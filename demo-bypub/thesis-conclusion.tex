% !TeX root = ./thesis.tex










%==============================
\chapter{Conclusion}
\label{chap:conclusion}

%-----
\EBlettrine{The} phenomenon known as collective animal behaviour is one of the most beautiful spectacles one can observe in nature. Although researched and analysed by scientists from many disciplines and perspectives it is still puzzling in many ways. Examples of such puzzling questions are why did collective behaviour (especially highly organized forms) evolve and why do we see so much variation in behaviour even in closely related species. Through the construction of a genetic fuzzy system capable of evolving various forms of collective behaviour this study represents an attempt in shedding additional light on potential reasons why highly organized collective behaviour evolved.

We started the study by expanding on a known fuzzy model \cite{lebarbajec2005fuzzy,lebarbajec2005simulating} for the purpose of studying predator-prey dynamics in hand-crafted models. Prey animats in this model \cite{demsar2014simulated} could exhibit two types of behaviour -- a social one where they actively strived for grouping (via cohesion and alignment drives) and an individualistic one where they did not. We focused on vision as the principal means of perception, took into account occlusion \cite{kunz2012simulations} and working memory limitations \cite{ballerini2008interaction,engle1999individual,sherry1989hippocampus}, and considered three target selection (predation) tactics. With the first tactic predators attacked the nearest out of visible prey individuals, with the second the most visually isolated prey individual out of the visible ones, and with the third the centre of the visible group (visible prey individuals). To achieve biological relevance we tuned the parameters based on realistic data about birds (starlings, \emph{Sturnus vulgaris}, for prey, and peregrine falcon, \emph{Falco peregrinus}, for the predator). The study suggests that the most successful predator is the one that attacks the most visually isolated individual, while the least successful predator is the one that attacks the centre of the visible group, a result similar to those reported by field observations \cite{zoratto2010aerial}. As a plus, results obtained by our study suggest that, from a prey individual's perspective, social behaviour is more advantageous than individualistic behaviour, which strengthens our belief in the hypothesis that cluster flocking might serve as be a mechanism for protection from predation.

Field observations suggest that predators in nature are able to, at least partially, overcome the defensive benefits of prey grouping by using an assortment of sophisticated hunting strategies \cite{cresswell2011predicting,forsman1998visual,gazda2005division,handegard2012dynamics,hector1986cooperative,kane2014falcons,lopez2006bottlenose,nottestad2002digging,rutz2012predator}. As the parametrization and tuning of such tactics in a hand-crafted model would be a tiresome task we developed an evolutionary model that simulates the evolution of composite target selection tactics \cite{demsar2015simulating}. The most successful predators were those that first dived deep into the centre of the nearby prey group causing chaos and dispersal of the group. Following that they targeted stragglers (individuals that in the process got separated from the rest of the group). The tactic, which we termed as the dispersing tactic, is similar in function to the tactics used by several predators in nature \cite{larsson2012why,pavlov2000patterns}. In our study predators that used the dispersing tactic came out as significantly more successful hunters in a direct competition with predators that used a mixture of simple tactics. Again a result corroborating field observations \cite{pavlov2000patterns}. However, this was true only in the case when our model took into account the possibility of predator confusion. The concept of predator confusion is based on the confusability hypothesis, which suggests that a group of visually similar prey might make it difficult for the predator to select and track its target \cite{nishimura2002predator,zheng2005behavior,kunz2006prey,olson2013predator,olson2016evolution,rutz2012predator}. A different story was the case of the prey's delayed response, a defensive manoeuvre where prey rather then escaping on first sight of the predator, delay their response to a later point in time, and then try to outsmart the predator with rapid movement \cite{partridge1982structure}. The only predators able to, at least to some degree, overcome the defensive benefits of this escape manoeuvre, were again the predators that used the dispersing tactic. Because the dispersing tactic yields higher success to predators we can assume that dispersing the group reduces the group's defensive benefits. This strengthens our belief in the hypothesis that compact groups of prey might function as a defensive mechanism from predation. The absence of an advantage of the dispersing tactic over simple predation tactics when predator confusion is not at play indicates that predator confusion might have played an important role in the evolution of advanced predation tactics, as well. All of these findings were a clear indication of potential interplay between target selection tactics and the evolution of prey group behaviour.

Several studies already pursued the artificial evolution of collective animal behaviour, most by tuning parameters of previously presented non-evolutionary models. Very few succeeded to evolve it from scratch, and even in these cases the evolved behaviour can be termed as ``crude.'' Based on presented material the successful studies portray either clumping \cite{biswas2014causes,hein2015evolution,witkowski2016emergence}, or swarming with collisions \cite{olson2013predator,olson2015exploring,olson2016evolution,witkowski2016emergence}. To study how predation tactics influence the evolution of prey behaviour we designed a novel open-ended, artificial life-like evolutionary model where the drives of individual animats are encoded via linguistic fuzzy rules \cite{demsar2017evolution}. In our genetic fuzzy system prey and predator animats coexist in a shared environment. Based on knowledge about predator target selection tactics gained from our previous research \cite{demsar2014simulated,demsar2015simulating} we designed several types of hand-crafted predators that attack evolving prey. Subsequently, in our model only the survival instincts of prey animats steer the evolution of their behaviour and collective behaviour will emerge only if it will help prey animats survive. We analysed the evolved prey behaviour and showed that based on biologically relevant metrics \cite{couzin2002collective,vicsek2012collective,tunstrom2013collective} our evolutionary model is capable of producing a wide range of behaviours, some qualitatively and visually similar to those reported by experimental studies \cite{tunstrom2013collective}. Since we used a genetic fuzzy system we were able to further analyse the evolved behaviours by studying the fuzzy rule bases that govern the actions of individual animats. Doing so we showed that when clustering the rule bases by the type of evolved behaviour and observing the average proportion of rule antecedents that contain predator related linguistic variables there exists a statistically significant difference between the rule bases. This gives us confidence in advocating that artificial life-like evolutionary modelling based on linguistic fuzzy rule-based systems could be used for answering the illusive biological questions ``why'' collective animal behaviour evolved, and due to their linguistic nature also provide a deeper insight into the ``how.''

To gain further insight into potential ``whys'' we used the newly developed genetic fuzzy system in a controlled experiment where prey evolved while subject to multiple, systematically picked predation tactics simultaneously \cite{demsar2016balanced}. The predation tactics can be split into two groups; those for which the natural defensive response of prey might be grouping and those for which the natural response might be dispersing. We classified the evolved behaviours using quantitative metrics in a similar fashion as previous studies \cite{couzin2002collective,vicsek2012collective,tunstrom2013collective}. When prey evolved while exposed to predators that adopted tactics from only one group the results of evolution corroborated with previous studies \cite{biswas2014causes,olson2013predator,olson2016evolution,wood2007evolving}; prey animats evolved either grouping or dispersing behaviour, with values of metrics characteristic for milling or swarming. When prey animats evolved while exposed to antagonistic pressures that at the same time steered the evolution towards grouping and towards dispersing we detected a significant increase of polarization in motion of prey groups. This suggest that exposure to antagonistic predation pressures might be a necessary requirement for prey individuals to evolve parallel movement. This could indicate that the direction of evolution (grouping or dispersing) is not A versus B, but a labile result -- whether grouping or dispersing evolves depends on a) the nature of the group, and b) the pressures that the group finds itself facing.

\paragraph{Limitations of this study and future work} Throughout our research we devised a number of ideas which could potentially lead to interesting future studies of collective animal behaviour. When it comes to application of evolutionary models for help with providing answers to biological questions the most obvious research advances lie in upgrades towards a higher biological relevance. Evolutionary models are usually simplified due to high computational demands of genetic algorithms and as a result the models are most often restricted to two dimensions, animats in them have unrealistic perception systems, and animats traditionally move with constant speeds, etc. To allow the animats to vary their speed in an evolutionary model we would probably need to implement some kind of fatigue system as well, so that, just like in nature \cite{norin2016measurement,roche2013finding}, animats would not be able to move with their maximum speed indefinitely. 

Another possible direction would be the investigation of how heterogeneity influences the evolved behaviour. Some recent studies suggest that in an algorithm mimicking artificial evolution heterogeneous groups might evolve a different behaviour than homogeneous ones \cite{olson2015exploring}. Others suggest that heterogeneous groups might be necessary to achieve a more ``natural'' behaviour \cite{demsar2013family}, and that differences among individuals might be essential for group coordination \cite{marras2012information,marras2013schooling}. In nature heterogeneity (both in behaviour as well as in physiology) is always present, for example birds in a flock often differ in size, gender, age, and some times even species \cite{lebarbajec2009organized,jolles2013heterogeneous}. It is not uncommon that stronger members of the group are positioned at the ``safer'' parts of it \cite{hamilton1971geometry}, which leaves the weaker individuals more exposed to predator attacks. Some predators often intentionally target weak prey individuals \cite{domenici2014howsailfish,marras2015notsofast}, and by using models that consider also physiological heterogeneity one could, apart from studying its influence on prey behaviour, also study how it influences the adaptation of predator target selection tactics.

To our knowledge, in most of the existing models \cite{demsar2014simulated,demsar2015simulating,demsar2016balanced,demsar2017evolution,nishimura2002predator,zheng2005behavior}, the predator animat, once it selects its target, uses classical pursuit \cite{nahin2012chases} to chase its target. In nature some predators use advanced pursuit tactics, for example some species of falcons use the technique of motion camouflage \cite{kane2014falcons}. With this technique they either camouflage themselves against a fixed background object so that the targeted individual observes no relative motion between them and the fixed object, or they approach the targeted individual in a way that, from the targeted individual's point of view the predators always appear to be on the same bearing \cite{justh2006steering}. One possible future study might therefore be a genetic fuzzy system for the evolution of predator pursuit tactics. Or co-evolution of prey behaviour and predator pursuit tactics.

In nature predators often resort to group hunting \cite{creel1995communal,escobedo2014groupsize,fanshawe1993factors,lett2004continuous,muro2011wolfpack,packer1988evolution,scheel1991group}. Occasionally they cooperate (\ie cooperative hunting) to increase the probability of a successful hunting event \cite{creel1995communal,packer1988evolution}. Even though in our current model prey animats are often attacked by several predators at once the predators do not cooperate in any way. Studying the evolution of predator cooperation during hunting events would probably also lead to an interesting study.

Another possible upgrade would be the consideration of short term memory, which comes to play when, for example a predator moves out of view of the targeted individual (out of range or in its blind area). In current models, the targeted individual completely forgets that the predator was attacking it just a moment ago.

An important question is also flight initiation distance. In certain fish species prey as a defence mechanism delay their response \cite{partridge1982structure}. Our research already showed, that a delayed response is quite effective with certain target selection tactics. With an evolutionary model, however, we can study under what conditions (if) such a delayed reaction will emerge. Research in this this direction is already on its way, our current provisional results suggest that the answer might be related to the ratio between predator and prey speed.

The rule base is probably the most important part of a fuzzy animat since it defines the drives of the animat, which have the highest influence on the animat's behaviour. In genetic rule learning the data base of a fuzzy system is static, it does not evolve. As our genetic fuzzy system executes genetic rule learning, fuzzy variables, the linguistic terms, and the interpretation of logic connectives, which are all defined in the data base of a genetic fuzzy systems were hand-crafted. In our research this did not appear to be a limitation as our genetic fuzzy system is capable of evolving many of the forms of collective behaviour that can be commonly observed in nature.

The degree of truth for each fuzzy term is defined by its membership function, these functions come in many shapes (triangular, trapezoidal, singleton, Gaussian, etc.). Even though our algorithm supports many different types of membership functions we developed all models by means of trapezoidal/triangular functions only. These provide the lowest ratio between computational complexity and ease of conceptualization, visualization, and explanation. Again, in the case of our genetic fuzzy system the use of triangular functions did not seem to be a limiting factor as the repertoire of evolved collective behaviours is wider than in previous similar studies \cite{biswas2014causes,hein2015evolution,olson2013predator,olson2015exploring,olson2016evolution,reynolds1993evolved,sayers2009evolved,spector2003emergence,wood2007evolving}. Nevertheless, the evolution of rule bases with more sophisticated types of membership functions and the evolution of the whole fuzzy knowledge base seem like promising research directions for our future work.

To conclude, evolutionary models allowed us to untangle a number of interesting riddles related to collective behaviour already, but judging by the current trends we believe that the best is yet to come.