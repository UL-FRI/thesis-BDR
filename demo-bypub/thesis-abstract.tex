% !TeX root = ./thesis.tex

% write thesis abstracts
% use \begin{abstract}\end{abstract} for abstract in english
% use \begin{povzetek}\end{povzetek} for abstract in slovene
% note
% with english as primary language the order is povzetek, abstract
% with slovene as primary language the order is abstract, povzetek



%==============================
\begin{povzetek}

  Skupinsko vedenje je fascinantno področje, ki preučuje dinamiko gibanja velikih skupin podobnih si entitet. Verjetno najbolj pogosta hipoteza glede skupinskega vedenja živali pravi, da združevanje in gibanje živali v skupinah deluje kot obrambni mehanizem pred plenilci.

  V tej disertaciji smo za namene raziskovanja te in sorodnih hipotez razvili več tehnik računalniškega modeliranja. V prvem delu smo najprej razširili obstoječi mehki model za računalniško simulacijo letenja ptic v jati, v katerega smo dodali plenilce ter nadgradili realizem pri simuliranju vida. Nato smo implementirali tri taktike napada, ki temeljijo na vizualnem zaznavanju plenilca in omejitvah kratkotrajnega delovnega spomina. Naši rezultati kažejo, da plenilec potrebuje dalj časa za ulov, če napada plen, ki se zadržuje v skupinah, kot če napada plen, ki ga združevanje v skupine ne zanima. Iz stališča plenilca je, pri napadu na plen, ki se zadržuje v skupinah, najboljša taktika napad najbolj vizualno izoliranega izmed vidnih posameznikov. Če pa se plen ne združuje v skupine, je najboljša taktika napad najbližje vidne tarče.

  V naslednji fazi doktorske disertacije smo razvili računalniški model, v katerem lahko plenilci, s pomočjo simulirane evolucije, prilagajajo ročno konstruirane taktike napada ter tako povečujejo svojo uspešnost. Zanimala nas je optimalna taktika napada, ko se plen zadržuje v skupinah ter optimalna taktika napada, če plen na napad reagira z zakasnjenim odzivom. Pri zakasnjenem odzivu posamezen plen z bežanjem ne začne takoj ko plenilca opazi, ampak z odzivom počaka in nato poskuša zbežati s pomočjo hitrih in ostrih zavojev. Pristop lahko predstavlja neke vrste obrambni mehanizem, saj lahko zaradi razlike v velikosti plen običajno izvaja hitrejše in ostrejše zavoje kot plenilec. Sodeč po naših simulacijah je napredna taktika, ki smo jo poimenovali razpršilna, iz stališča plenilca najuspešnejša. Pri tej taktiki se plenilec najprej zapodi v sredino skupine plena, da jo razprši. Ko se to zgodi, plenilec napada izolirane posameznike. Simulacije z zakasnjenim odzivom plena nakazujejo, da plen lahko zniža učinkovitost plenilcev z uporabo tega obrambnega mehanizma. Pri tem je bila razpršilna taktika edina, s katero so lahko plenilci delno izničili prednosti, ki jih plenu nudi zakasnjeni odziv.

  S pomočjo pridobljenega znanja smo nato razvili mehki evolucijski model primeren za simuliranje evolucije skupinskega vedenja. Osredotočili smo se na simuliranje evolucije skupinskega vedenja plena, če je le-ta podvržen pritiskom s strani večjega števila plenilcev, ki pri lovu uporabljajo različne ročno načrtovane taktike napada. Pokazali smo, da se s pomočjo našega modela lahko razvije več različnih režimov skupinskega vedenja, ki so sorodni tistim, ki jih lahko opazimo v naravi (rojenje, kroženje okrog praznega jedra, usklajeno gibanje, dinamično gibanje). Ko smo pri analizi naborov pravil opazovali delež pravil, ki upošteva informacijo o plenilcih, smo opazili statistično signifikantno razliko med nabori pravil, ki se skrivajo za različnimi režimi skupinskega vedenja. Ta ugotovitev nakazuje na to, da taktike napadov, ki jih uporabljajo plenilci, verjetno vplivajo na smer evolucije vedenja plena.

  V zadnjem delu disertacije smo s sistematično izbiro ročno načrtovanih taktik napada analizirali, kako razni pritiski plenilcev vplivajo na evolucijo vedenja plena. Taktike napada, ki smo jih pri tem uporabili, lahko razvrstimo v dve skupini. V prvi so tiste, pred katerimi se plen lahko ubrani z združevanjem v skupine, v drugi pa tiste, pri katerih je za plen bolje, če se ne zadržuje v skupinah. V naši raziskavi je plen razvil usklajeno dinamično gibanje zgolj v primerih, ko se je razvijal, ob prisotnosti taktik iz obeh skupin sočasno in nikoli, ko so bile prisotne zgolj taktike iz ene same skupine. Rezultat nakazuje, da so konfliktni pritiski možen predpogoj za razvoj usklajenega gibanja skupin.

\end{povzetek}

%==============================
\begin{abstract}

  Collective behaviour is a fascinating field that studies coordinated motion of large groups of similar entities. Probably the most common hypothesis about the origins of collective animal behaviour suggests that it might function as a defensive mechanism against predation.

  In this thesis we used various computational techniques to study this hypothesis. We started by expanding an existing fuzzy model for the computer simulation of bird flocking with predators and visual perception. We implemented three target selection tactics that take into account the visual perspective of the predator (attack the nearest visible individual, attack the most visually isolated individual, and attack the centre of the visible group). Our results suggest that for prey individuals social behaviour (governed by the separation, alignment and cohesion drives) as opposed to individualistic (governed exclusively by the separation drive) is the most beneficial (predators take longer to capture their target). Predators, on the other hand, capture social prey individuals quicker when they attack the most visually isolated individual, but capture individualistic prey faster if they focus on the nearest prey individual.

  In the next stage we developed an evolutionary model for tuning hand-crafted composite predator attack/target selection tactics. For reasons of computational simplicity we here expanded on a known mathematical model of prey collective behaviour. This allowed us to concentrate on predator target selection tactics. We investigated the evolution of the optimal tactic with respect to prey behaving collectively and prey that performed a delayed response. With the latter prey individuals instead of responding immediately at the first sight of the predator delay the response to a later point in time and then try to outsmart the predator by performing rapid twists and turns. This might be an advantageous defensive manoeuvre because prey can remain in a compact group for as long as possible and because prey individuals are usually smaller than predators and as such have a higher turn rate. Our results suggest that a composite tactic termed dispersing tactic, where the predator first dives deep into the group of prey and then targets the most peripheral individual, is the best tactic. Experiments with prey's delayed response suggest that prey individuals can indeed increase their survivability by using this defensive manoeuvre and the dispersing tactics seems to be the only tactic capable of at least partially diminishing the effectiveness of the preys' delayed response. This was a clear indication of potential interplay between target selection tactics and prey behaviour.

  Armed with this knowledge, we developed an artificial life-like open-ended evolutionary model, where the behaviour of prey and predator individuals is governed by fuzzy logic. In this model we focused on the evolution of prey behaviour when prey individuals face different predation tactics. We demonstrated that in this model prey individuals evolve different types of collective behaviour (swarm, milling, polarized, dynamic). Interestingly, the analysis of the evolved rule bases showed a statistically significant difference between different types of behaviour in the proportion of rules that take into account predator related information. This suggested that the predation pressures the prey are subject to during evolution might have an influence on the behaviour that evolves.

  Our last step of research was thus a controlled experiment where prey evolve under various predation tactics. Here we let prey evolve under four predation tactics, two of which according to previous research pressure prey to evolve dispersing and two pressure prey to evolve grouping. Our results suggest that antagonism in predation pressures, where prey are exposed to predation pressures for which the best response is both grouping and dispersing simultaneously, might be necessary for prey to evolve polarized movement.

\end{abstract}