% !TeX root = ./thesis.tex










%==============================
\chapter{Review of published work}
\label{chap:review}

%-----
\EBlettrine{This} thesis can be divided into four separate stages. In the first two stages, \emph{Simulated predator attacks on flocks: a comparison of tactics}, and \emph{Simulating predator attacks on schools: evolving composite tactics}, we studied how various predation tactics influence the survivability of prey and what kind of adaptations by prey individuals decrease the predation success of predators. In the third stage, \emph{Evolution of collective behaviour in an artificial world using linguistic fuzzy rule-based systems}, we developed a genetic fuzzy system that is capable of simulating artificial evolution and thus generating a number of diverse forms of collective behaviour observable in nature. In the last stage, \emph{A balanced mixture of antagonistic pressures promotes the evolution of parallel movement}, we systematically investigated how various predation tactics influence the evolved behaviour of prey.

%-----
\section[Simulated predator attacks on flocks: a comparison of tactics]{Simulated predator attacks on flocks:\\ a comparison of tactics}

Starting with an existing fuzzy logic based model \cite{lebarbajec2005fuzzy,lebarbajec2005simulating} we integrated recent discoveries about vision based interaction (topologic interaction \cite{ballerini2008interaction} and occlusion \cite{kunz2012simulations}) and introduced vision based predators \cite{demsar2014simulated}. This allowed us to study the behaviour of systems that contain several types of animats including predator-prey interaction. In most individual-based models that simulated predator-prey interactions at the time of that study, predators attacked the centre of prey groups. This appears contradictory to the hypothesis that collective behaviour evolved as a protection against predators. For this reason we investigated how predator attack tactics and prey behaviour influence the survivability of prey individuals. We tuned the parameters that define the movement characteristics of animats, \eg speed and manoeuvrability, on empirical data so that the group dynamics and escape patterns were similar to those in nature.

Our results suggest that prey individuals that exhibit social behaviour (governed by the separation, alignment and cohesion drives) have a higher chance of surviving predator attacks opposed to prey individuals with individualistic behaviour (governed exclusively by the separation drive). The results support the hypothesis that grouping might function as a defensive mechanism. By implementing three target selection tactics that take into account the visual perspective of the predator (attack the nearest visible individual, attack the most visually isolated individual, and attack the centre of the visible group) we were able to provide support for this hypothesis from the predator's perspective as well. When predators attacked social prey individuals, they captured their targets faster if they attacked the most visually isolated individual, which suggest that moving in tight and evenly spaced groups might make it hard for predators to select and track their targets. When predators attacked prey with individualistic behaviour they were the most successful if they focused on the nearest prey individual. The reason might be because in the absence of social behaviour predators can reach the nearest prey individual the fastest. In nature this tactic is typically used by predators that try to minimize energy costs for prey capture.

%-----
\section[Simulating predator attacks on schools: evolving composite tactics]{Simulating predator attacks on schools:\\ evolving composite tactics}

In previous individual-based models of predator-prey interactions, including ours, predators mainly used one of several basic attack tactics; attack the centre of the group/the most central individual, attack the nearest individual, or attack the most (visually) isolated individual. In nature however, predators appear to use elaborate target selection and pursuit/hunting tactics \cite{cresswell2011predicting,forsman1998visual,gazda2005division,handegard2012dynamics,hector1986cooperative,kane2014falcons,lopez2006bottlenose,nottestad2002digging,rutz2012predator} and prey vice-versa resort to different defensive tactics (\eg grouping, grouping with a delayed response). In addition, results of our first stage of research suggested that the predator's optimal tactic depends on the prey's behaviour. This motivated us into developing an evolutionary model that tunes the parameters of hand-crafted predators \cite{demsar2015simulating}. For reasons of comparability and computational simplicity we opted to expand on an existing hand-crafted non-fuzzy prey and predator model. In the developed model predators were able to adapt their target selection tactic to diminish the effectiveness of the defensive actions of prey and increase their hunting efficiency.

Our results suggest that the basic attack tactics (attack the most peripheral prey individual, attack the nearest or attack the most central prey individual) are suboptimal, as they were all outperformed by a composite attack tactic, which we named as the \emph{dispersing tactic}. With the dispersing tactic the predators first dived deep into the centre of a nearby group of prey and then after causing chaos and dispersion of the group focused on isolated individuals. Interestingly this tactic can be commonly seen in nature when various predators, \eg swordfish (\emph{Xiphias gladius}), attack groups of prey \cite{larsson2012why,pavlov2000patterns}. In turn our results corroborate with the hypothesis that late but rapid group escape patterns commonly observed in nature \cite{partridge1982structure} help prey individuals decrease the efficiency of predators. Our results also suggest that when predators attack groups of prey that delay their escape response the dispersing tactic is again the most successful one from the predator's perspective. The dispersing tactic seems to be the only tactic capable of at least partially diminishing the effectiveness of the preys' delayed response.

During this study we also investigated the impact of the confusion hypothesis on the evolution of composite predation tactics. The confusion hypothesis suggests that prey groups work as a defensive mechanism because predators have a hard time tracking a single target in a vast group of visually similar animals. Because without consideration of predator confusion all tactics converged to similar values, our findings seem to suggest that predator confusion might have played an important role in the evolution of composite predation tactics, as well. The results of this study were a clear indication of potential interplay between target selection tactics and prey group behaviour.

%-----
\section{Evolution of collective behaviour in an artificial world\\ using linguistic fuzzy rule-based systems}

Armed with the knowledge gained in the first two stages of this research we developed an artificial life-like, open-ended, evolutionary model where competition between predators and prey in the battle for survival is the principal force that steers the evolution of prey \cite{demsar2017evolution}. In this model the behaviour of prey and predator individuals was governed by fuzzy logic. We used the model to study the evolution of prey behaviour when prey individuals are forced to live in a shared environment with various types of predators using diverse predation tactics. The predators were hand-crafted and tuned based on results from the previous stages of this research.

We demonstrated that the newly developed model is capable of producing a number of different forms of collective behaviour that both visually and quantitatively \cite{couzin2002collective,vicsek2012collective,tunstrom2013collective} resemble collective motion commonly observed in nature (swarming, milling, polarized motion, dynamic motion). Since the behaviour of every individual prey animat was described in the form of linguistic if-then rules this allowed us to study the logic behind the evolved behaviours. Interestingly, the analysis of the evolved rule bases showed a statistically significant difference between different forms of collective behaviour in the proportion of rules that take into account predator related information. This suggests that the predation pressures the prey are subject to during evolution might have an influence on the behaviour that evolves.

%-----
\section{A balanced mixture of antagonistic pressures promotes\\ the evolution of parallel movement}

Based on the indication that predation pressure might influence the form of evolved collective behaviour the last stage of research was a controlled experiment where prey evolved while subject to multiple simultaneous predation pressures. We investigated the influence of four predation tactics, two of which according to previous research pressure prey to evolve dispersing and two that pressure prey to evolve grouping. We analyzed the evolved behaviour via the prey density, polarization, and angular momentum metrics \cite{couzin2002collective,olson2016evolution,tunstrom2013collective}.

Experiments with predators to which the expected natural defensive response was either grouping or dispersing corroborate previous studies \cite{biswas2014causes,olson2013predator,olson2016evolution,wood2007evolving}. When predators pressure prey towards grouping, prey evolve behaviours that result in an increase in prey density. When they pressure prey towards dispersing, prey evolve behaviours that result in a decrease in prey density. In all of these cases prey most often resorted to collective motion similar to swarming and milling. More interesting results came from experiments where prey evolved while under threat from predators that use antagonistic predation pressures. Pressures that push prey to evolve grouping and dispersing simultaneously. Our results suggest that antagonism in pressures, where prey are exposed to pressures for which there is no clear best response (grouping or dispersing), might be necessary for prey to evolve polarized movement.

