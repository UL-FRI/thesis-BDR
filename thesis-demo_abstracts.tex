% write thesis abstracts
% use \begin{abstract}\end{abstract} for abstract in english
% use \begin{povzetek}\end{povzetek} for abstract in slovene









%==============================
\begin{Povzetek}

%-----
V pričujoči disertaciji se posvečam vprašanju, kako uporabiti mehko logiko za modeliranje letenja ptic v jati. V raziskavo me je poneslo prepričanje, da bi bilo z uporabo preprostih lingvističnih opisov (t.j. množice lingvističnih pravil) to kompleksno dinamiko mnogo laže definirati kot z uporabo matematičnih enačb. Večina obstoječega znanja o obnašanju jat ptic je namreč na voljo le v obliki opisov in razlag opažanj, ki so jih ornitologi postavili na podlagi svojih terenskih izkušenj.

V svojem delu formaliziram razširjeni Moorov avtomat, ki ga poimenujem animat, ter pokažem, da lahko v to obliko prevedem tudi najodmevnejše dosedanje modele letenja ptic v jati. V nadaljevanju omogočim določanje pravil obnašanja animata z uporabo lingvističnih pravil mehke logike ter njegovo uporabnost prikažem s postavitvijo mehkega modela za simulacijo letenja ptic v jati, pri čemer pa se omejim le na dvodimenzionalni prostor brez ovir. Del disertacije posvetim tudi definiciji metrik za kvalitativno ocenjevanje modelov letenja ptic v jati ter izvedem primerjavo najodmevnejših modelov z novo vpeljanim mehkim modelom.

V prihodnosti kot prvi nadaljni korak vidim nadgradnjo mehkega modela do delovanja v tridimenizionalnem prostoru z ovirami. V smislu modeliranja jat ptic pa imam kasneje namen vpeljati tudi območja hranjenja ter nadgraditi model mehke digitalne ptice s
potrebo po hrani. Nenazadnje pa v prihodnosti vidim tudi možnost uporabe mehkega animata za postavitev mehkega modela za simuliranjo obnašanja množic v paniki, kot je to na primer pri izbruhu požara v zgradbi, s čimer bi se odprla možnost študije ustreznosti evakuacijskih poti.
\end{Povzetek}









%==============================
\begin{Abstract}

%-----
One of the most obvious observations that can be made about the natural world is that it is based on dynamics. Indeed the latter is as present in nature (fluids, animal groups, \etc) as it is in man-made systems (vehicle traffic, autonomous robots, \etc). In most cases we talk about cooperation or coexistence of groups of entities, which can be homogeneous or not. Furthermore it can be said that in most cases the basic means of cooperation or coexistence is coordinated movement. In man-made systems this capability had to be introduced, but in the natural world it has been present since the days of its creation. This is why most authors when modelling, simulating or introducing coordinated movement capabilities into men-made systems draw their inspiration from the natural world -- primarily from coordinated groups of moving animals. Examples of coordinated animal groups abound, but the most commonly known are flocks of birds, schools of fish and swarms of insects.

The scientific study of animal behaviour in general is called ethology. Until recently it has been practiced entirely through observation of living animals and their interactions with the environment. Nevertheless, knowledge gained while modelling and simulating cooperating and coexisting systems presents a new approach to the study of animal behaviour. Appearing is indeed the possibility to study animal behaviour through computer modelling and simulation, in other words, to study animal behaviour by developing digital (or simulated) animals living and coexisting in a digital universe and observing them while cooperating with each other. Ethologists can thus move from observing behaviour in the natural world, where the displayed behaviour depends on preconditions on which one has hardly any influence, to observing behaviour in the digital world, which is fully under the scientist's control. Furthermore, the approach is gaining appeal because of the ever increasing processing and visualisation capabilities of personal computers. In the future the ethologist will thus first construct a digital counterpart of the animal of their interest and then observe the simulated behaviour inside a perfectly controllable environment. In doing so they will be able to test the existing or forming new hypotheses about `why' and `how' animals behave as they do.

Of all coordinated groups of moving vertebrates, birds are at the same time the easiest to observe and perhaps the most difficult to study. With this in mind the bibliography about bird flocking was reviewed, where emphasis was given to the studies regarding computer modelling and simulation. The review showed that the research field was at its peak at the end of 1980s, when the two most influential models were developed. Afterwards research interest slowly subsided. A thorough review established that the primary reason for the latter might be because the models usually employ complex mathematical methods and are as such difficult to understand. This is further emphasized by the syntactical confusion, which is present in most of the research papers discussing them. 

To help remedy that and allow a uniform approach to modelling the dynamics of organized groups of moving animals, a formal definition of the animat was developed. The latter was based on a reformulation of the transition function of the Moore automaton. In order to diminish syntactical confusion, the animat was then used to reproduce the two most influential models for the computer simulation of bird flocking.

Furthermore, as the existing models are difficult to understand and/or use by the audience they were designed for, primarily because they employ complex mathematical methods, fuzzy logic was introduced into the animat and a formal definition of the fuzzy animat was presented. The latter was then used to construct a new fuzzy model for the computer simulation of bird flocking. Results from the analysis and comparison of the fuzzy model with the existing models showed that comparable and in some aspects more `natural' behaviour can be obtained by using the fuzzy model even when basing it on the common sense knowledge about the behaviour of flocking birds.
\end{Abstract}
