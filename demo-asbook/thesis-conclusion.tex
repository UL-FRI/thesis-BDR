%signal path for graphics files
\graphicspath{{img/}}





%==============================
\chapter{Conclusion}
\label{chap:conclusion}


%----
\section{Principal Scientific Contributions}
In this section I briefly summarize the principal scientific contributions. With each contribution I list the dissertation sections where the topic is discussed. In addition I also list the references to the publications that discuss the topic and of which I was the first author or co-author. Note that the listed references were presented on international conference meetings or published in internationally renowned scientific journals and were thus internationally reviewed and discussed.
\begin{itemize}
%
\item \emph{Design and formal definition of an extended Moore automaton (animat)}\\
So as to achieve syntactical clarity in future studies of the dynamics of organized groups of moving animals a formal definition of the animat was introduced. The latter is an extension of the Moore automaton, taking into account a certain set of animal characteristics (\ie\ perception, drives, action selection). The animat can be used to construct a digital (or simulated) animal and thus, by using a collection of animats, the study of the dynamics of a group of moving animals is made possible. Drafts of the animat definition were published in \cite{lebar_bajec:2002,lebar_bajec:2003a,lebar_bajec:2003b,mraz:2004}.
%
\item \emph{Design and formal definition of a fuzzy extended Moore automaton (fuzzy animat)}\\
In Chapter~\ref{chap:fuzzyAnimat}, to simplify the construction of digital animals, fuzziness was introduced into the animat and the fuzzy animat was presented. The latter enables the construction of digital animals using ambiguous (uncertain, vague, \etc) knowledge and data. With this the transition from linguistic description to mathematical formulae that is required in the traditional approach is omitted. Drafts of the fuzzy animat definition were published in \cite{lebar_bajec:2003a,lebar_bajec:2003b,mraz:2004}.
%
\item \emph{Application of the fuzzy animat to the problem of the simulation of bird flocking}\\
The usability of the fuzzy animat is shown in section~\ref{sec:fuzzyAnimat:afd} by applying it to the study case of modelling bird flocking. The design of the fuzzy digital bird was based on common knowledge about the behaviour of real birds. Drafts of the presented model were published in \cite{lebar_bajec:2003a,lebar_bajec:2003b,lebar_bajec:2005a}.
%
\item \emph{Design and formal definition of a set of metrics used for comparing the simulation results from different computer models of bird flocking}\\
In an attempt to promote an analytical comparison of different models section~\ref{sec:analysis:metrics} presents a set of metrics that allow the comparison of the simulation results obtained using different computer models of bird flocking. The metrics are such that, in the case when the required data is available, they allow also a comparison of the simulated results to the dynamics of a real flock. Drafts of the presented metrics were published in \cite{lebar_bajec:2002,lebar_bajec:2003a,lebar_bajec:2003b,lebar_bajec:2005a}.
%
\item \emph{Comparison and analysis of the simulation results obtained by using different computer models of bird flocking}\\
Section~\ref{sec:analysis:comparison} utilizes the metrics that were introduced in section~\ref{sec:analysis:metrics} to compare and analyse the simulation results obtained by using the fuzzy model with those obtained by using Reynolds's \cite{reynolds:1987,reynolds:1999} model. Analogous to other fields of modelling \cite{mraz:1999,mraz:2001}, the analysis shows that by using simple fuzzy logic rules, for which no type of rule fitting or learning \cite{bonarini:1996} has ever been used, comparable and in a way more `natural' results can be achieved. Drafts of the presented results were published in \cite{lebar_bajec:2002,lebar_bajec:2003a,lebar_bajec:2003b,lebar_bajec:2005a}.
%
\end{itemize}

%----
\section{Future Research Directions}
As already discussed in the dissertation, my foremost step in future research directions will be to extend the fuzzy model for a computer simulation of bird flocking to three dimensions. This will require rewriting the knowledge-bases for all of the fuzzy animat's fuzzy drives. Another important step in the future research of the topic is modelling inaccurate visual perception while taking into account the binocular overlap. 

More distant steps are the inclusion of environmental obstacles in the digital universe and modelling obstacle avoidance. This will definitely have a great impact on the overall behaviour. In my opinion the simple fuzzy weighted sum action selection function will no longer be satisfactory and new improved techniques will have to be developed. Good starting points are the prioritized action selection proposed by Reynolds \cite{reynolds:1987} and the multiple objective action selection proposed by Pirjanian \cite{pirjanian:1998}.

Nevertheless, even in the current state of the fuzzy model, it would be interesting to use it as the input for the physics-based method for synthesis of bird flight animations, developed by Wu and Popović \cite{wu:2003}. I assume that by combining my model and their musculoskeletal structure and animation method a completely different and in my opinion highly natural behaviour could be achieved. This is primarily because their method would complement my behaviour model with locomotoric constraints that have a substantial effect on the displayed behaviour. \sidenote{Unfortunately this is still highly impractical because the method used by Wu and Popović is very time-consuming and thus not suitable for real-time simulation.}{v1.1.20050210 [FHH]: but possible for non-real-time, rendered applications?} Nevertheless it would be possible to use it for a non-real-time, rendered application.

To conclude, I believe that the animat construction framework is very suitable as a bottom-up approach to modelling quantum dot cellular automata \cite{walus:2004}. The first drafts of research on this subject were presented in \cite{mraz:2004} as well as in the recent BSc thesis by P. Pečar, who is currently preparing an extended article about the topic.
