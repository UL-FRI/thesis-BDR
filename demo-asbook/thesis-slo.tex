% write your main thesis in logical individual chapters
% for better organization use a separate folder for images 
\graphicspath{{img/}}









%==============================
%\appendix
%\makeatletter
% \if@slovene\else 
%  \ifxetex\selectlanguage{slovenian}\else\selectlanguage{slovene}\fi
% \fi
%\makeatother
\begin{slovenian}

\chapter{Razširjeni povzetek}
\label{ch:povzetek}


%-----
\section{Motivacija}
Eno očitnejših dejstev o resničnem svetu je, da temelji na dinamiki. Opazujemo jo namreč lahko tako v naravnem okolju (živi in neživi svet) kot tudi v umetno zgrajenih sistemih (letalski in cestni promet, skupine robotov itd.). V večini primerov govorimo o sodelovanju oziroma sobivanju skupin osnovnih entitet, ki so lahko homogene ali ne. V večini primerov lahko tudi rečemo, da je najbolj osnovno sodelovanje entitet njihovo usklajeno gibanje. Umetno zgrajenim sistemom je bilo to sposobnost treba vdahniti, nasprotno pa je v naravnem svetu prisotna vse od njegovega nastanka. Prav zato večina avtorjev \cite{bonabeau:1999, braitenberg:1984,jadbabaie:2003,mataric:1995} pri modeliranju, simuliranju in uvajanju sposobnosti usklajenega gibanja v umetno zgrajene sisteme črpa navdih prav iz naravnega okolja -- primarno s preučevanjem obnašanja živali, za katere je značilno usklajeno gibanje. Narava je namreč polna primerov usklajenega gibanja, pri čemer so najbolj tipični primeri jate ptic, trume rib, roji insektov itd.

Znanstvena veda, ki obravnava obnašanje živali v splošnem, je etologija. Do nedavnega je temeljila predvsem na opazovanju živali v naravnem okolju ter njihove interakcije z njim \cite{watts:1998}. Dandanes pa znanje, pridobljeno pri modeliranju in simuliranju sodelujočih sistemov, ponuja nove pristope tudi k obravnavi obnašanja živali. Poraja se namreč možnost obravnave s pomočjo računalniških modelov in simulacij, ali povedano drugače, možnost obravnave na osnovi izgradnje digitalnih (simuliranih) živali, ki živijo in sobivajo v digitalnem svetu, ter opazovanja, kako nanj vplivajo in v njem sodelujejo \cite{bentley:2002}. Etologom se tako ponuja možnost, da preidejo z opazovanja obnašanja v naravnem okolju, kjer je ponovljivost prikazanega obnašanja pogojena z začetnimi pogoji, na katere lahko le redkokdaj vplivamo, na opazovanje obnašanja v digitalnem okolju, ki je pod popolnim nadzorom znanstvenika. Tak pristop postaja vse bolj zanimiv predvsem zaradi zviševanja procesnih in predstavitvenih zmožnosti osebnih računalnikov. Etolog bodočnosti bo torej najprej razvil digitalni primerek živali, ki je predmet njegovih raziskav, in zatem opazoval simulirano obnašanje v popolnoma obvladljivem okolju. Tako bo lahko preizkušal obstoječe ali gradil nove hipoteze o tem, `kako' in `zakaj' se živali obnašajo tako, kot se.

%-----
\section{Jate ptic}
Od vseh vretenčarjev, ki živijo v skupinah in se usklajeno gibajo, je ptice najlažje opazovati, a istočasno verjetno tudi najtežje obravnavati \cite{heppner:1997}. Za razliko od ptic lahko namreč za večino sesalcev privzamemo, da se gibljejo po ravnini, kar močno olajša analiziranje dinamike njihovega gibanja. Ribe pa lahko, po drugi strani, zapremo v akvarij, s čimer omejimo prostor gibanja ter analizo dinamike zopet močno olajšamo. Prav slednje je verjetno eden vodilnih vzrokov, da številna dela obravnavajo ravno trume rib \cite{aoki:1982,dill:1997,mcfarland:1997,partridge:1982,shaw:1962,terzopoulos:1994,tu:1994,tu:1999,ward:2001,zaera:1996}. Kot že rečeno, so pravo nasprotje temu ptice. Te se namreč gibljejo znotraj tridimenzionalnega prostora in to s hitrostmi, ki so lahko zelo velike. Prostor, po katerem se gibljejo, je torej težko, če ne skoraj nemogoče, omejiti. Obravnavo dinamike gibanja jat ptic pa poleg omenjenega otežuje še dejstvo, da so preleti jat v naravi nepredvidljivi. Namreč tudi v primerih, ko z dovolj visoko statistično verjetnostjo vemo, kje se pojavljajo, ne moremo napovedati njihovega gibanja. Pridobivanje in analiza podatkov o naravnih prosto letečih jatah ptic sta tako izredno težavna, kar je razvidno tudi iz literature \cite{gould:1974,heppner:1997,jaffe:1997,moyle:1998}.

Prav zato smo se usmerili na pregled literature o jatah ptic. Največji poudarek smo pri tem namenili raziskavam računalniškega modeliranja in simulacije letenja ptic v jati. Pregled je pokazal, da je največ tovrstnih raziskav nastalo konec 80. let minulega stoletja. Takrat sta bila namreč razvita tudi najodmevnejša modela, ki so ju postavili Reynolds \cite{reynolds:1987} ter Heppner in Grenander \cite{heppner:1990}. 

%-----
\section{Simuliranje jat ptic}
Reynolds, strokovnjak za računalniško grafiko, se je s problemom soočil predvsem zaradi težavnosti ročnega animiranja jat ptic za potrebe filmske industrije. Do tedaj je slednje temeljilo na mukotrpnem ročnem postavljanju ključnih pozicij posameznih članov jate. Reynolds pa je pod vplivom raziskovalnega področja \emph{umetnega življenja} (\eng{artificial life}) iskal rešitev, kako bi s pomočjo računalniškega modela in simulacije animiranje jate ptic avtomatiziral. 

Nasprotno sta Heppnerja, kot ornitologa, v načrtovanje računalniškega modela vodili predvsem ključni vprašanji `kako' in `zakaj'. Povedano natančneje sta to vprašanji \emph{``Kako so razmeroma majhne ptice sposobne leteti v tako velikih jatah ter izvajati hitre nenadne zavoje brez medsebojnih trkov?''} in \emph{``Zakaj velike ptice letijo v tako natančnih razporeditvah?''}. Težavnost pridobivanja podatkov ga je prisilila v navezo z matematikom Grenandrom ter iskanje drugačnih pristopov k obravnavi dinamike gibanja jate ptic. 

Ne glede na različni raziskovalni področji imata oba modela skupna izhodišča. V obeh primerih gre za predpostavko, da dinamika jate ne temelji na centralnem viru procesiranja odločitev, temveč na porazdeljenem procesiranju. O smeri in hitrosti gibanja se tako odloča vsak član jate posamezno. Procesiranje je pri tem odvisno od stanja okolice posameznega člana ter tendenc, ki izvirajo iz njegovih teženj. Poleg tega pa so v obeh primerih težnje izvedene s pomočjo matematičnih enačb (geometrijski izračuni, diferencialne enačbe itd.), do katerih so avtorji prišli z eksperimentalnim delom. Po Reynoldsu \cite{reynolds:1987,reynolds:1999} so težnje:
\begin{itemize}
\item \emph{razmik} (\eng{separation}): član jate skuša držati pravšnjo oddaljenost od svojega sosedstva (ostalih bližnjih članov jate), 
\item \emph{usmerjenost} (\eng{alignment}): član jate skuša svojo smer in hitrost izenačiti s smerjo in hitrostjo svojega sosedstva in 
\item \emph{vezljivost} (\eng{cohesion}): član jate se skuša usmerjati v središče svojega sosedstva.
\end{itemize}
Po Heppnerju in Grenandru \cite{heppner:1990} pa:
\begin{itemize}
\item \emph{vračanje} (\eng{homing}): član jate se skuša zadrževati v bližini počivališča oziroma območja hranjenja (\eng{roosting area}),
\item \emph{uravnavanje hitrosti} (\eng{velocity regulation}): član jate skuša leteti z vnaprej določeno hitrostjo in se ob morebitni spremembi k tej hitrosti vrniti ter
\item \emph{interakcija} (\eng{interaction}): če sta dva člana jate preblizu drug drugemu, se skušata oddaljiti; če sta preveč oddaljena, drug na drugega ne vplivata, v nasprotnem primeru pa se skušata drug drugemu približati.
\end{itemize}
Poleg teh treh teženj pa sta Heppner in Grenander v svoji želji po čimvečji verodostojnosti modela s Poissonovim stohastičnim procesom simulirala tudi vplive različnih nepredvidljivh motenj, kot so sunki vetra. V svojem delu \cite{heppner:1990} priznavata, da jima brez slednjega ne bi uspelo dobiti zadovoljivega obnašanja.

%-----
\section{Animat}
Gledano splošneje segajo prvi poskusi modeliranja umetnega življenja nazaj vse do leta 1940. Tedaj je namreč John von Neumann postavil osnove definicije strukture, imenovane \emph{celularni avtomat} (\eng{cellular automaton}), na kateri temelji večina obstoječih modelov umetnega življenja \cite{adami:1998,bonabeau:1999,emmenche:1994,gardner:1970,langton:1984,rucker:1993}. 

Celularni avtomat je definiran kot prostor celic, katerih osnovna značilnost je notranje stanje. Skozi diskretne časovne korake lahko posamezna celica stanje spreminja, in sicer na osnovi lastnega stanja v predhodnem koraku ter stanj, ki so jih v predhodnem koraku imele njej sosednje celice. V večini primerov se kot model celice uporablja \emph{Moorov avtomat} (\eng{Moore automaton}) \cite{mraz:2000}. žal zahteva po nespremenljivi razporeditvi celic vpliva na uporabnost celularnega avtomata kot pristopa k modeliranju dinamike gibanja organiziranih skupin. 

V disertaciji smo se zato osredotočili na Moorov avtomat ter ga nadgradili tako, da lahko z množico avtomatov predstavimo digitalni svet, sestavljen tako iz živih bitij kot tudi neživih predmetov in katerega predstavitev ni pogojena z njihovo prostorsko razporeditvijo. Ker osnovna vodila pri nadgradnji vsebinsko sovpadajo s pojmom \emph{animat} (\eng{animat}), ki ga je brez formalizacije vpeljal Wilson \cite{wilson:1985}, a se dandanes kljub temu uporablja za poimenovanje razreda računalniško simuliranih živih bitij in robotov \cite{cliff:1993,watts:1998}, smo naš razširjeni Moorov avtomat poimenovali animat. 

Razširitev se pri tem nanaša predvsem na funkcijo prehajanja stanj, ki je v razširjeni obliki predstavljena kot trinivojska funkcija. Prvi nivo je namenjen modeliranju zaznavanja okolja, drugi nivo modeliranju teženj, ki vplivajo na obnašanje modeliranega bitja, tretji pa modeliranju razvrščanja in združevanja akcij, ki uresničujejo težnje modeliranega bitja. Trinivojska funkcija prehajanja stanj nam tako omogoča modeliranje na osnovi enega izmed osnovnih mišljenj o načinu delovanja živih bitij, kjer je končna akcija posledica procesa zaznavanja prisotnosti impulzov iz okolja ter uresničevanja lastnih ciljev.

Kot prikaz uporabnosti animata smo slednjega uporabili za postavitev obeh najodmevnejših računalniških modelov za simulacijo letenja ptic v jati ter pri tem opazili, da imata modela precej skupnih točk. Oba namreč temeljita na zakonitostih privlačnosti in odbojnosti, katerih pomembnost za notranjo strukturo in dinamiko organizirane skupine je predstavil že Okubo \cite{okubo:1980}. Kljub temu pa sta modela zaradi uporabe matematičnih postopkov (geometrijski izračuni, diferencialne enačbe itd.) težko obvladljiva avditoriju, ki sta mu pravzaprav namenjena (biologi, etologi, behavioristi itd.).

%-----
\section{Mehka logika}
\emph{Mehko modeliranje} (\eng{fuzzy modelling}) je postalo aktualno s hitrim povečevanjem procesorskih zmogljivosti. Temelji na \emph{mehki logiki} (\eng{fuzzy logic}), ki je nastala kot naravna posledica vpeljave teorije \emph{mehkih množic} (\eng{fuzzy sets}). Slednjo je v svojem članku leta 1965 prvič predstavil Lotfi A. Zadeh \cite{zadeh:1965}. Teorija mehkih množic predstavlja posplošitev teorije klasičnih (ostrih) množic z vpeljavo delne pripadnosti. Osnovna razlika med klasično in mehko množico tako temelji na obravnavi pripadnosti elementa množici. Klasični množici lahko element le pripada ali ne pripada, mehki pa lahko pripada tudi delno. Torej, če pripadnost elementa opazovani množici ovrednotimo na osnovi \emph{pripadnostne funkcije} (\eng{membership function}), zaloga vrednosti omenjene funkcije, ki jo pri klasični množici predstavlja množica $\left\{0,1\right\}$, postane v primeru mehke množice zvezni interval $\left[0,1\right]$.

Temelje modeliranja na osnovi dvomnih vedenj in robnih pogojev je postavil Witold Pedrycz \cite{pedrycz:1993}. V aktualni literaturi je moč najti mnogo konkretnih zgledov mehkega modeliranja (modeliranje požarov v naravnem okolju \cite{mraz:1999,vakalis:2004a,vakalis:2004b}, modeliranje snežnih plazov \cite{barpi:2004}, modeliranje krmiljenja bele tehnike \cite{mraz:2001} itd.), pri čemer le redki sodijo v področje masivnih dinamičnih procesov. Mehkega modeliranja gibanja ptic v jati po načelih teženj pa v literaturi ni moč zaslediti. 

%-----
\section{Mehki animat}
Modeliranje dinamike gibanja organiziranih skupin je zahtevna naloga, ki potrebuje natančno poznavanje obnašanja modeliranega bitja. Žal pa je natančno poznavanje pogosto nedosegljivo. Slednje pomeni, da so nam v večini primerov na voljo le opisi in razlage opazovanega obnašanja, ki so običajno pod močnim vplivom opazovalca. To pomeni tudi, da je prenos takšnega dvoumnega znanja v matematične enačbe težak in pogostokrat za avditorij, ki se z obnašanjem živali ukvarja, skoraj nemogoč. 

Prav slednje nas je vodilo v nadgradnjo animata z vpeljavo mehkosti. \emph{Mehki animat} (\eng{fuzzy animat}) je tako postal struktura, ki omogoča postavitev modela na osnovi \emph{dvomnih} (\eng{ambiguous, uncertain, vague} itd.) znanj ali vedenj ter dvomnih vhodnih podatkov. Temeljna prednost mehkega animata je možnost neposrednega lingvističnega opisa (programiranja) teženj modeliranega bitja ter procesiranje na osnovi dvomnih vhodnih podatkov. 

Mehki animat smo uporabili za modeliranje posamezne ptice v jati, pri čemer smo se omejili na dvodimenzionalni prostor brez ovir. Pri modeliranju teženj naše digitalne ptice smo izhajali iz osnovnega splošnega vedenja o obnašanju ptic v jati.

%-----
\section{Rezultati}
Z namenom analitične in kontrolirane eksperimentalne primerjave obstoječih modelov z našim mehkim modelom smo postavili nabor metrik, ki omogočajo objektivno primerjavo. Pri konkretni primerjavi smo se osredotočili predvsem na primerjavo našega mehkega modela z modelom, ki ga je postavil Reynolds \cite{reynolds:1987,reynolds:1999}. Analiza je pokazala, da smo, tako kot na drugih področjih modeliranja \cite{mraz:1999,mraz:2001}, z uporabo preprostih mehkih pravil, ki temeljijo na splošnem vedenju in za katere niso bili uporabljeni postopki učenja \cite{bonarini:1996}, dobili primerljivo ter v nekaterih pogledih bolj `naravno' obnašanje.

%-----
\section{Zaključek}
V pričujoči doktorski disertaciji so tako podani naslednji izvirni prispevki k znanosti:
\begin{itemize}
\item \emph{postavitev in formalizacija razširjenega Moorovega avtomata (animata),}
\item \emph{postavitev in formalizacija mehke oblike razširjenega Moorovega avtomata (mehkega animata),}
\item \emph{uporaba modela mehke oblike razširjenega Moorovega avtomata (mehkega animata) za potrebe realizacije modela gibanja ptic v jati,}
\item \emph{postavitev in formalizacija metrik za primerjavo simulacijskih rezultatov različnih modelov gibanja ptic v jati ter} 
\item \emph{primerjava in analiza simulacijskih rezultatov različnih modelov gibanja ptic v jati.}
\end{itemize}

\end{slovenian}