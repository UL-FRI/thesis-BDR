% load FRI thesis document class with options: 
% - english for writing in english
% - slovene for writing in slovene
% - [PhD, funded] for a funded PhD thesis - doktorska disertacija MR iz gospodarstva
% - press for print version
% - online for online version - coloured and active figure/table/... links
\documentclass[english,PhD]{FRIthesis}

\author[A={I Lebar Bajec}]{Iztok Lebar Bajec}

\title[language=english]{Fuzzy model for a computer simulation of bird flocking}
\title[language=slovene]{Mehki model za računalniško simulacijo letenja ptic v jati}

\keywords[language=english]{bird, flock, boid, animat, fuzzy logic, fuzzy modelling, fuzzy animat, artificial life, behavioural animation}
\keywords[language=slovene]{ptica, jata, boid, animat, mehka logika, mehko modeliranje, mehki animat, umetno življenje, vedenjska animacija}

% primer v slovenskem jeziku
%\companyLogo{cc/company_logo-slo}
%\approvedBy[id=1, title={izredni profesor za računalništvo in informatiko}, role={mentor in član ocenjevalne komisije}]{dr. Nikolaj Zimic}
%\approvedBy[id=2, title={izredni profesor za računalništvo in informatiko}, role={predsednik ocenjevalne komisije}]{dr. Blaž Zupan}
%\approvedBy[id=2, affiliation={University of Rhode Island}, title={Professor Emeritus of Biological Sciences}, role={zunanji član ocenjevalne komisije}]{dr. Frank H. Heppner}
%\companyLogo{cc/company_logo-eng}
\approvedBy[title={Associate Professor of Computer and Information Science}, role={advisor and examiner}]{dr. Nikolaj Zimic}
\approvedBy[title={Associate Professor of Computer and Information Science}, role={examiner}]{dr. Blaž Zupan}
\approvedBy[affiliation={University of Rhode Island}, title={Professor Emeritus of Biological Sciences}, role={external examiner}]{dr. Frank H. Heppner}

\previousPublication{%
	Lebar~Bajec I, Zimic N, Mraz M (2003) 
	Fuzzifying the thoughts of animats in
	\emph{Fuzzy Sets and Systems: Proceedings of the 10th International Fuzzy Systems Association World Congress (IFSA 2003)}, 
	Lecture Notes in Artificial Intelligence, Vol. 2715,
	eds. Bilgiç T, De~Baets B, Kaynak O. 
	(Springer-Verlag, Berlin), 
	doi:\,\doi{10.1007/3-540-44967-1_23}.
}
\previousPublication{%
	Lebar~Bajec I, Zimic N, Mraz M (2003) 
	Boids with a fuzzy way of thinking, in
	\emph{Proceedings of Artificial Intelligence and Soft Computing (ASC 2003)},
	ed. Leung H.
	(ACTA Press, Anaheim), pp. 58--62.
}
\previousPublication{%
	Lebar~Bajec I (15/04/2005) 
	\emph{Fuzzy logic and bird flock simulations}. 
	Invited lecture, University of Rhode Island, Department of Biological Sciences, Kingston, RI.
}
\previousPublication{%
	Lebar~Bajec I, Zimic N, Mraz M (2005) 
	Simulating flocks on the wing: the fuzzy approach.
	\emph{Journal of Theoretical Biology}
	doi:\,\doi{10.1016/j.jtbi.2004.10.003}.
}

\forceDate{6}{2005}

% define cover 
% \cover[title=<title to appear on cover page>, loc=<location of the title>, colour=<colour of the title>]{coverImage}
% defaults: title=\title{#}, loc=NE, colour=P1797
% notes:
%  title - use title to insert prefered line breaking with '\\' 
%  loc - options NE, SE, NW, SW
%  colour - options P1797, white, black
\cover[title={Fuzzy model for a computer simulation of bird flocking}, loc={NE}, colour={P1797}]{./cc/coverImage.jpg} 
% define spine
% \spine[title=<title to appear on spine>, loc=<location of the title>, colour=<colour of the title>]{thesisId in hexadecimal}
% defaults: title=\title[A={#}]
% notes:
%  title - use title to provide a shortended title for the spine 
%  pages - privde the total number of pages
\spine[title={Fuzzy model for a computer simulation of bird flocking}, pages=134]{80}

% define common abbreviations
% define i.e.
\newcommand{\ie}{i.e.}
% define e.g.
\newcommand{\eg}{e.g.}
% define etc.
\newcommand{\etc}{etc.}
% define et al.
\newcommand{\etal}{et al.}
% define tab.
\newcommand{\tab}{Tab.}
% define eq.
\newcommand{\eq}{Eq.}
% define fig.
\newcommand{\fig}{Fig.}
% define figs.
\newcommand{\figs}{Figs.}
% define QED symbol
\newcommand{\QED}{\hfill\ensuremath{\Box}}

% define macros for frequently used commands 
\newcommand{\eng}[1]{angl. #1}

% define custom macros for notation
%% define real numbers symbol
\newcommand{\Rset}{\ensuremath{\mathbb{R}}} 
\newcommand{\R}{\Rset} 
%% define natural numbers symbol
\newcommand{\Nset}{\ensuremath{\mathbb{N}}} 
\newcommand{\N}{\Nset} 
%% define euclidean vector space symbol
\newcommand{\Eset}{\ensuremath{\mathbb{E}}} 
\newcommand{\E}{\Eset} 
%% define degrees symbol
\renewcommand{\deg}{°} 
\newcommand{\degC}{{\deg}\ensuremath{\textnormal{C}}} 
%
%% define set style
\newcommand{\set}[1]{{\ensuremath{\mathbfup{#1}}}} 
% \symbfup resolves issue where \tilde{\mathbf{}} produces an "Internal error: bad native font flag in `map_char_to_glyph'"
% see https://tex.stackexchange.com/questions/353443/xetex-unicode-math-tilde-on-bold-characters
%% define power set
\newcommand{\powset}[1]{{\ensuremath{\mathcal{P}(#1)}}} 
%% define vector style
\newcommand{\vect}[1]{{\ensuremath{\mathbfup{#1}}}}
% define automaton style
\newcommand{\autom}[1]{{\ensuremath{\mathup{#1}}}} 
% define fov symbol
\newcommand{\fov}{fov}
% define algebraic product symbol
\newcommand{\aprod}{\ifxetex\ensuremath{\capdot}\else\ensuremath{\mathaccent\cdot\cap}\fi}
% define equation reference style
\providecommand{\eqref}[1]{(\ref{#1})} 

% declare signum operator
\DeclareMathOperator*{\sgn}{sgn}
% declare centre of gravity operator
\DeclareMathOperator*{\cog}{cog}

% define fuzzy automaton style
\newcommand{\fautom}[1]{{\ensuremath{\tilde{\mathup{#1}}}}} 
% define fuzzy set style
\newcommand{\fset}[1]{{\ensuremath{\tilde{\set{#1}}}}} 
% define fuzzy power set
\newcommand{\fpowset}[1]{{\ensuremath{\mathcal{F}(#1)}}} 
% define fuzzy function style
\newcommand{\ffunc}[1]{{\ensuremath{\tilde{#1}}}}
% define fuzzy rule style
\newcommand{\frule}[1]{\textsc{\MakeLowercase{#1}}}
% define keyword style
\newcommand{\kwd}[1]{\textsc{#1}}
% define fuzzy variable style
\newcommand{\fvar}[1]{\emph{#1}} 
% define fuzzy value style
\newcommand{\fval}[1]{\emph{#1}}
% define logical statement style
\newcommand{\statement}[1]{\ifxetex{\ensuremath{\mathscr{#1}}}\else{\ensuremath{\mathppl{#1}}}\fi}

\makeatletter

\show\bibsection
\show\thebibliography
\show\endthebibliography
\show\@bibsetup

%\let\oldbib\
%\def\thebibliography#1{
%	\newcounter{longestlabel}
%	\expandafter\setcounter{longestlabel}{#1*10}
%	\typeout{\arabic{longestlabel}^^J}
%	\bibsection 
%	\parindent \z@ 
%	\bibpreamble 
%	\bibfont 
%	\list {\@biblabel {\the \c@NAT@ctr }}{\@bibsetup {\arabic{longestlabel}}\global \c@NAT@ctr \z@ }
%	\ifNAT@openbib 
%		\renewcommand \newblock {\par }
%	\else 
%		\renewcommand \newblock {\hskip .11em \@plus .33em \@minus .07em}
%	\fi 
%	\sloppy 
%	\clubpenalty 4000
%	\widowpenalty 4000 
%	\sfcode `\.\@m 
%	\let \NAT@bibitem@first@sw \@firstoftwo 
%	\let \citeN \cite 
%	\let \shortcite \cite 
%	\let \citeasnoun \cite
%}
%\def\endthebilbiography{
%	\bibitem@fin 
%	\bibpostamble 
%	\def \@noitemerr {\PackageWarning {natbib}{Empty `thebibliography' environment}}
%	\endlist
%	\bibcleanup
%}

\makeatother

\begin{document}

% TODO: show as an example of subfigure use
%\begin{figure}%[b]
%	\begin{subfigure}[b]{.5\figurewidth-3.5pt}
%		\noindent\hrulefill\par
%		\includegraphics{img/fig[fuzzySet]a}
%		\caption{sub A}\label{labA}
%	\end{subfigure}
%	\hfill
%	\begin{subfigure}[b]{.5\figurewidth-3.5pt}
%		\noindent\hrulefill\par
%		\includegraphics{img/fig[fuzzySet]b}
%		\caption{sub B}\label{labB}
%	\end{subfigure}
%	\caption{Membership functions of a crisp set of real numbers `from 3 to 5' \subref{labA} and a fuzzy set of real numbers `close to 4' \subref{labB}.}
%	\label{fig:fuzzy:set0}
%\end{figure}

%\makecoverpage % auto-generate cover.tex file and build using xelatex

\frontmatter
 	\maketitle
	\makeapprovedby
	\makepreviouspublication
 	%signal path for graphics files
\graphicspath{{img/}}





%==============================
\cleardoublepage
\thispagestyle{empty}

\vspace*{55pt}

\begin{centering}

{\textsc{in memoriam}
 \par}

\vspace{.5cm}

{Damjan Oseli 
 \par}

{1975--2004
 \par}

\end{centering}

\vfill






%==============================
\cleardoublepage
\thispagestyle{empty}

\vspace*{55pt}

\begin{centering}

{\setbox0=\hbox{\emph{``Things never turn out the way you think they will.''}}
 \begin{minipage}{\wd0}
 \emph{``Things never turn out the way you think they will.''}\\%[6pt]
 \null \hfill  --- Michael Crichton, \emph{Prey}, 2002.
 \end{minipage}
 \par}

\vspace{.5cm}

{\setbox0=\hbox{\includegraphics{pixar[forTheBirds]}}
 \begin{minipage}{\wd0}
 \includegraphics{pixar[forTheBirds]}\\%[6pt]
 \null \hfill  --- Pixar, \emph{For the Birds}, 2000.
 \end{minipage}
 \par}

\end{centering}

\vfill
\cleardoublepage

 	% !TeX root = ./thesis.tex

% write thesis abstracts
% use \begin{abstract}\end{abstract} for abstract in english
% use \begin{povzetek}\end{povzetek} for abstract in slovene
% note
% with english as primary language the order is povzetek, abstract
% with slovene as primary language the order is abstract, povzetek


%==============================
\begin{povzetek}

%-----
\lipsum[2-4]

\end{povzetek}


%==============================
\begin{abstract}

%-----
\lipsum[2-4]

\end{abstract}

 	% Do thank those that have helped you









%==============================
\begin{acknowledgements}

%-----
Firstly, I wish to thank my thesis advisor, Associate Professor Nikolaj Zimic, without whose support this dissertation would not have been possible. Furthermore, I sincerely thank Assistant Professor Miha Mraz for the long and some times stressing discussions, which always (one way or another) resulted in my moving forward.

My special thanks go to Professor Frank H. Heppner for a prompt response to my e-mail requesting access to his publications and for later generously agreeing to serve as external examiner of my dissertation, despite his busy schedule and the long distance he had to travel.

I thank all members of the Computer Structures and Systems Laboratory who were forced to listen to my discussions related to bird flocks even in such delicate times as coffee breaks.

Last but not least, I wish to thank my beloved Maja for taking her time and reading through the manuscript and correcting all typographical errors and inserting the missing or deleting the misleading commas; TVTB.
\end{acknowledgements}
 	\tableofcontents 

\mainmatter
 	%signal path for graphics files
\graphicspath{{img/}}







%==============================
\chapter{Introduction}
\label{chap:introduction}


%-----
\section{Motivation}
With the increase of the processing and presentational capabilities of personal computers, the field of computer modelling and simulation has been gaining research interest even in the areas that a decade ago did not believe that modelling and simulation were suitable for them. One such area is the modelling of the dynamics of organized groups of moving animals. Some of the typical examples of such groups are pedestrians \cite{brogan:1998,helbing:1995}, bird flocks and fish schools \cite{aoki:1982,heppner:1990,lebar_bajec:2002,lebar_bajec:2003a,lebar_bajec:2003b,lebar_bajec:2005a,okubo:1986,reynolds:1987,terzopoulos:1994,tu:1994,tu:1999,zaera:1996}, ant colonies \cite{bonabeau:1999,resnick:1997}, \etc\ For the majority of these groups their dynamics is not based on a centralized source of decision processing, but on a distributed one. Each individual member of the group thus processes its own decision, depending on the perceived state of the universe and the tendencies that originate from the drives guiding it. 

If I focus on bird flocks: the first and most influential models were developed by Reynolds \cite{reynolds:1987} and Heppner and Grenander \cite{heppner:1990}. In both cases the drives are implemented through mathematical equations (geometrical calculations, differential equa\-ti\-ons, \etc), which the authors obtained by means of trial-and-error experimentation. According to Reynolds \cite{reynolds:1987,reynolds:1999}, these drives are:

\begin{itemize}
\item \emph{separation}: each member of the flock tries to maintain a certain separation distance from its flockmates (nearby flock members),
\item \emph{alignment}: each member of the flock tries to match its flight speed and flight direction with that of its flockmates, 
\item \emph{cohesion}: each member of the flock tries to fly toward the centre of its flockmates.
\end{itemize}

According to Heppner and Grenander \cite{heppner:1990}, the drives are:

\begin{itemize}
\item \emph{homing}: each member of the flock tries to stay in the roosting area,
\item \emph{velocity regulation}: each member of the flock tries to fly with a certain predefined flight speed---it tries to return to that speed if perturbed,
\item \emph{interaction}: if two flockmates are too close to one another, they try to move apart; if they are too distant, they do not influence each other; otherwise they try to move closer together. 
\end{itemize}

However, in addition to these three drives, Heppner and Grenander modelled also the \emph{random impact}, which was intended to simulate the random distractions that are present in a natural environment (wind gusts, distractions from moving objects on the ground, \etc). They implemented it as a Poisson stochastic process and admitted that without its inclusion they were unable to produce flock-like behaviour. 

Upon the analysis of the relevant bibliography it has been determined that the existing models have some weaknesses, which can be briefly summarized as follows:

\begin{itemize}
\item \emph{syntactical confusion}: most of the authors do not present formal definitions of the models; in the majority of cases the models are only described, which is usually not a good enough basis for an actual implementation; the latter requires a formal specification in the form of an automaton or algorithm; 
\item \emph{lack of evaluation metrics}: regardless of the numerous models, neither an analytical comparison between the obtained results nor their evaluation have been performed; in this sense the models have never been truth-tested;
\item \emph{usability}: most of the models are based on complex mathematical formalisms (differential equations, random processes, computation of the centre of mass, \etc); in this sense the models are difficult to understand and/or use by the audience they were designed for (biologists, ethologists, behaviourists, etc.).
\end{itemize}

Fuzzy modelling has gained momentum with the increase of processing capabilities. It is based on fuzzy logic, which emerged as an outgrowth of fuzzy set theory. The latter was first introduced in 1965 by Lotfi A. Zadeh \cite{zadeh:1965}. Fuzzy set theory is a generalization of conventional (or crisp) set theory by the introduction of the concept of partial membership. The main difference between the two is thus in the interpretation of membership. In conventional set theory, an object can be either a member of the observed set or not a member of the observed set; in fuzzy set theory, however, it can also be a partial member of the observed set. Therefore if the object's membership to the observed set is computed by means of a membership function, then in the case of a crisp set this function maps to the set $\left\{0,1\right\}$, whereas in the case of a fuzzy set it maps to the entire unit interval $\left[0,1\right]$. 

The basics of modelling using uncertain knowledge and preconditions were set by Witold Pedrycz \cite{pedrycz:1993}. In the current literature numerous examples of fuzzy modelling can be found (modelling of fire spread prediction \cite{mraz:1999,vakalis:2004a,vakalis:2004b}, modelling of snow ava\-lan\-ches \cite{barpi:2004}, modelling of the control of kitchen appliances \cite{mraz:2001}, \etc). However, only some of them model massive dynamic processes. Furthermore in the extensive literature I did not find any attempt at fuzzy modelling of bird flocking.


%----
\section{Scientific Contributions}
In the light of the preceding discussion about the existing models' weaknesses, the following scientific contributions are presented in this dissertation:

\begin{itemize}
\item \emph{design and formal definition of an extended Moore automaton (animat)}:
with respect to the current syntactical confusion I design and present a formal definition of an extended Moore automaton \cite{kohavi:1978}, which allows a uniform approach to modelling the dynamics of organized groups of moving entities; the ideas behind the above mentioned formal definition correspond with the term animat, which was introduced, without a formal definition, by Wilson \cite{wilson:1985}; in spite of that, in the last decade the term has become the synonym for a class of computer simulated animals or robots;
\item \emph{design and formal definition of a fuzzy extended Moore automaton (fuzzy animat)}:
afterwards I upgrade the extended Moore automaton with the introduction of fuzziness \cite{zadeh:1965}; the latter allows the construction and implementation of the automaton by using ambiguous (uncertain, vague, \etc) knowledge and data; one of the main advantages of fuzzy logic is its ability to permit a direct linguistic description (programming) of an arbitrary decision system;
\item \emph{application of the fuzzy animat to the problem of the simulation of bird flocking}:
to present its usability I employ the fuzzy animat to model a member of a bird flock; the model is limited to a two-dimensional space without obstacles; these two preconditions originate from comparable models of other authors; 
\item \emph{design and formal definition of a set of metrics used for comparing the simulation results from different computer models of bird flocking}:
the authors of the related models base the analysis of their simulation results mostly on visual grounds; the approach is subjective and is, as such, useless for an analytical comparison of different models; with this in mind I design and formally define a set of metrics that allow the comparison of the simulation results, and if the required data be available, perhaps also a comparison to the dynamics of a natural flock;
\item \emph{comparison of the simulation results obtained by using different computer models of bird flocking}:
I then use the introduced metrics to compare the simulation results obtained by using the newly introduced model with those obtained by using Reynolds's model \cite{reynolds:1987,reynolds:1999}; the latter represents the main reference for the majority of the existing models;
\item \emph{analysis of simulation results from different computer models of bird flocking}:
I analyse the differences in the simulation results of the compared models. 
\end{itemize}


%----
\section{Methodology}
In the quest to achieve the discussed scientific contributions, the following methodologies have been employed:

\begin{itemize}
\item review and analysis of the bibliography related to modelling massive dynamic processes; in doing so, I have learnt about the most important related models;  
\item application of the acquired knowledge to the design of an extended Moore automaton (animat);
\item application of the knowledge about fuzzy logic to the design of a fuzzy extended Moore automaton (fuzzy animat);
\item development of a computer application (simulator) for the simulation of bird flocking;
\item experimental work with the simulator and analysis of the simulation results;
\item comparison of the results obtained with the simulation and those obtained by using the computer models of other authors;
\item monitoring and review of all of the new activities in the related fields and active participation in the form of journal articles and conference contributions.
\end{itemize}


%----
\section{Dissertation Overview}
The main question to be addressed by this dissertation is, ``How to use fuzzy logic for modelling bird flocking?'' I feel that flock-like behaviour could be much more easily described by using simple linguistic descriptions (\eg\ collections of if-then rules) than by using mathematical equations. Indeed, the existing knowledge about the behaviour of flocks is usually available in the form of the observer's linguistic descriptions and explanations of the perceived behaviour.\footnote{``\emph{It seems likely that if a bird, say, to the left and in front of another bird turned suddenly in front of the trailing bird, the trailing bird would have time to react, and turn in the same direction, avoiding collision.}'' From personal correspondence with Frank H. Heppner.} The existing models were thus arrived at by approximating such linguistic descriptions using mathematical equations. Moreover, considerable amounts of advanced mathematical skills were required for this transition. Fuzzy logic \cite{zadeh:1965} is a very popular, successful and widespread approach for modelling processes, for example fire spread prediction \cite{mraz:1999,vakalis:2004a,vakalis:2004b}, which are too complex for classical mathematical methods. I feel that by using it to describe the simulated animal's behaviour the transition from the linguistic description to the actual behaviour could be made much shorter and much more understandable. In this way researchers involved with the study of the dynamics of organized groups of moving animals would gain a tool that enabled them to construct the subjects of their interest---the digital animals---and study their behaviour through simulations.

This dissertation presents the \emph{fuzzy animat construction framework} and through the study case of bird flocking also its usage. In Chapter~\ref{chap:birdFlocks} the bird flocking literature is reviewed and the two most influential models are presented. Chapter~\ref{chap:animat} presents a formal definition of an extended Moore automaton that can be used as a uniform approach to modelling the dynamics of organized groups of moving animals. Chapter~\ref{chap:fuzzyModelling} presents a brief overview of fuzzy logic and fuzzy modelling. In Chapter~\ref{chap:fuzzyAnimat} fuzzy logic is used to introduce fuzziness into the earlier formalized extended Moore automaton, which is then used to construct a fuzzy digital bird. Chapter~\ref{chap:analysis} is dedicated to the analysis of the dynamics of organized groups of moving fuzzy digital birds and Chapter~\ref{chap:conclusion} concludes by reviewing the achieved scientific contributions and presenting the future research directions.

%--
\section{Notation}
\sidenote{This dissertation assumes that the reader is familiar with the fundamentals of the theory of conventional (or crisp) sets and conventional (crisp or two-valued) logic. Furthermore, it is assumed that the reader is familiar with Euclidean vector spaces and corresponding vector operations.}{v1.1.20050210 [FHH]: OK, but biologists will not be able to follow along.} This section is included solely to introduce the notation that is employed, as needed, throughout the dissertation.

\begin{itemize}
\item $\N=\left\{1,2,3,\ldots\right\}$ the set of all positive natural numbers,
\item $\N_n=\left\{1,2,3,\ldots,n\right\}$ the set of all positive natural numbers lower or equal $n$,
\item $\R$ the set of all real numbers,
\item $\R^+$ the set of all non-negative real numbers,
\sidenote{\item $\E$ an Euclidean vector space such as $\R^2$, $\R^3$,}{ \href{http://mathworld.wolfram.com/EuclideanSpace.html}{EuclideanSpace}}
\item $\set{A},\ldots,\set{Z}$ conventional (or crisp) sets,
\item $\fset{A},\ldots,\fset{Z}$ fuzzy sets,
\item $\vect{a},\ldots,\vect{z}$ vectors from vector space $\E$,
\item $\langle x_1,x_2,\ldots,x_n \rangle$ ordered $n$-tuple of elements $x_1$, $x_2$, \ldots, $x_n$.
\end{itemize}

\noindent In addition
\begin{itemize}
\item $\mathrm{iff}$ is shorthand expression for ``if and only if'',
\item $\exists$ is shorthand expression for ``exists'',
\item $\forall$ is shorthand expression for ``for all'',
\item $\left|\set{A}\right|$ is shorthand for the size of set $\set{A}$,
\item $\mu_\set{A}$ is shorthand for the membership function of crisp set $\set{A}$,
\item $\powset{\set{A}}$ is shorthand for the family of all crisp sets that can be defined on universal set $\set{A}$, 
\item $\mu_\fset{A}$ is shorthand for the membership function of fuzzy set $\fset{A}$,
\item $\fpowset{\set{A}}$ is shorthand for the family of all fuzzy sets that can be defined on universal set $\set{A}$, 
\item $\left\|\vect{a}\right\|$ is shorthand for the size (norm) of vector $\vect{a}$,
\item $\vect{a}^0$ is shorthand for the normalized vector $\vect{a}$ (\ie\ a vector in the same direction as $\vect{a}$ but with size $1$),
\item $\lfloor\vect{a}\rceil^a$ is shorthand for the truncated vector $\vect{a}$ (\ie\ a vector in the same direction as $\vect{a}$ but with size lower or equal $a$).
\end{itemize}

 	%signal path for graphics files
\graphicspath{{img/}}





%==============================
\chapter{Bird Flocks}
\label{chap:birdFlocks}


%-----
\section{Flock Formations}
Of all coordinated groups of moving vertebrates, birds are at the same time the easiest to observe and perhaps the most difficult to study \cite{heppner:1997}. This is primarily because most of the animal congregation research is highly dependent on collecting \cite{heppner:1997,jaffe:1997} large sets of four-dimensional data (\ie\ three in space and one in time). In fact, as a contrast to birds, most mammals move in a two-dimensional plane, which simplifies obtaining real-world data, and fish can be brought into a laboratory and enclosed in an aquarium for study. Probably because of the easier and more fruitful tracking of confined objects \cite{heppner:1997,parrish:1997a}, fish schools have been a frequent research theme \cite{aoki:1982,dill:1997,mcfarland:1997,partridge:1982,shaw:1962,terzopoulos:1994,tu:1994,tu:1999,ward:2001,zaera:1996}. 

On the other hand, scientists involved in bird congregations research are challenged by the highly difficult and almost luck-dependent data collection \cite{heppner:1997}. Just the fact that a single bird in an organized flock can move through six degrees of freedom at velocities up to 150 km/h makes collecting real-world data very difficult. Flocks fly in a three-dimensional space that cannot be easily contained and their flight paths cannot be predicted. Even by knowing the locations where there is a reasonable probability of flock appearance one cannot predict its flight path. Furthermore the three-dimensional acquisition and analysis techniques generally demand either fixed camera or detector positions. The free-flying flocks must thus be either induced to fly in the field of the cameras or the cameras must be placed in locations where there is reasonable probability that adventitious flocks will move through their field. The difficulty of data acquisition is evident even from the existing literature \cite{gould:1974,heppner:1997,jaffe:1997,moyle:1998}. According to Heppner \cite{heppner:1997}, this may be one of the reasons why there is a current of imaginative speculation, and lively controversy in literature on bird congregation structure and internal dynamics, but little data. However, regardless of the difficulty of data acquisition, some basic understanding of bird flocks is already available. 

Birds can fly in disorganized groups, such as gulls orbiting over a landfill, or organized groups, such as the vees of waterfowl. To the evolutionist, behaviourist, or ecologist, any group is of interest, but nevertheless organized groups raise most questions. In his pioneering work from 1974 Heppner \cite{heppner:1974a} presented the first definitions of the two groups. 

\begin{definition}
\label{def:aggregation}
A disorganized group of birds or \emph{flight aggregation} is a group of flying birds lacking coordination in turning, spacing, velocity, flight direction of individual birds and time of take-off or landing, assembled in a given area.
\end{definition}

\begin{definition}
\label{def:flock}
An organized group of birds or \emph{flight flock} is a group of flying birds, coordinated in one or more of the following parameters of flight: turning, spacing, velocity, and flight direction of individual birds, and time of take-off and landing.  
\end{definition}

In this study Heppner also presented a classification of flight flock formations and a discussion of the leading research directions.\footnote{In this dissertation I primarily consider groups of birds in flight and thus will be concerned neither with their take-off nor landing. For reasons of clarity the term flight in flight flock and flight aggregation will therefore be omitted.} According to his study, there are two major classes of flight formations: \emph{line formations} and \emph{cluster formations}. Their characteristics had and still have a substantial influence on the leading research directions. Today \cite{heppner:1997,parrish:1997a}, as it was then \cite{heppner:1974a}, the examination of bird flocks is still led by two primary questions. The first, usually expressed while observing a skein of geese flying overhead, is ``\emph{Why do they fly in such a precise alignment?}'' The second comes to mind when we observe 5000 European Starlings, \emph{Sturnus vulgaris}, turning and wheeling over a roost. We ask ourselves ``\emph{How do they achieve such coordination and polarity?}'' The question `why' is thus usually expressed in reference to relatively large birds, like waterfowl, flying in line formations, whereas the question `how' is in reference to relatively large flocks of small birds, like sandpipers, flying in cluster formations. 

%--
\subsection{Line Formations}
Line formations (\fig~\ref{fig:lineFormations}) are groups of relatively large birds, such as waterfowl and pelicans, flying in a single line, or joined single lines. Typically they are approximately two-dimensional and show a rather high degree of regularity in spacing and alignment. In line formations birds fly in a single line, one behind the other (\emph{column}), one beside the other ({\em front\/}) or staggered stepwise from the bird at the head of the formation (\emph{echelon}). In nature left and right echelons can be found and 
%\marginpar{\footnotesize Birds flying in globular clusters generally fly in apparent close order and can be seen making very rapid turns. Similarly, front clusters tend to have very precise spacing and turning.} 
frequently a left echelon becomes a right echelon, and vice versa. However, the transition is not a swing from side to side, but rather a temporary breakup of the formation. In line formations birds also fly in joined single lines; left and right echelons joined at the tip of the formation (\emph{`J'} and \emph{`V'}) or at the tail of the formation (\emph{inverted `J'} and \emph{inverted `V'}). In the `V' and inverted `V' formation the left and right echelon are approximately the same size, whereas in the `J' and inverted `J' formation one is considerably larger.

\begin{figure}%[htb]
\includegraphics{fig[lineFormations]}
\caption{Line formations: column, front, echelon, `J', `V', inverted `J', and inverted `V' \cite{heppner:1974a}.}
\label{fig:lineFormations}
\end{figure}

%--
\subsection{Cluster Formations}
Cluster formations (\fig~\ref{fig:clusterFormations}) are relatively large flocks of small birds, like sandpipers, characterized by development in the third di\-men\-si\-on, and rapid, apparently syn\-chro\-nous turns. In cluster formations birds are typically distributed over a three-di\-men\-si\-onal space of an irregular spheroidal shape that is, when observed perpendicularly to the plane of flight: as wide as it is long ({\em globular cluster\/}), wider than it is longer ({\em front cluster\/}), or longer than it is wider ({\em extended cluster\/}). Birds flying in globular clusters generally fly in apparent close order and can be seen making very rapid turns. Similarly, front clusters tend to have very precise spacing and turning. The front cluster is often seen in pigeons. However, birds flying in extended clusters tend to be rather disorganized, with frequent breakoffs and shifts of position. This formation may simply be a disorganized group of birds that happen to be flying independently toward a common destination (\ie\ an aggregation) \cite{heppner:1974a}.

\begin{figure}%[htb]
\includegraphics{fig[clusterFormations].pdf}
\caption{Cluster formations: front cluster, globular cluster, and extended cluster \cite{heppner:1974a}.}
\label{fig:clusterFormations}
\end{figure}


%-----
\section{Simulating Bird Flocks}
While observing line formations, one is impressed by the precision with which relatively small numbers of large birds maintain themselves in accurate spatial alignment and angular orientation with their flockmates. On the other hand, while observing cluster formations, the attention is drawn to the coordination that enables large numbers of small birds, flying in close order, to wheel and turn without suffering mid-air collisions. In the first case the primary interest is the functional significance of formation flight \cite{heppner:1997,speakman:1998}. In the second the attention is given to the synchrony, or apparent synchrony, in the turning movements and the necessity or presence of a leader guiding these manoeuvres.

In the mid 1980s different papers appeared, suggesting that coordination in cluster flocks might be achieved by the application of the mathematics of nonlinear dynamics \cite{okubo:1986} and that flocking might be an emergent property arising from individuals following simple rules of movement \cite{heppner:1987}. At the same time, but working in another field of study, namely computer graphics, Reynolds \cite{reynolds:1987} published a ground-breaking seminal paper that first presented a computer model of bird flocking. His primary objective was a believable animation of a bird flock. In his study a collection of individuals whose behaviour is governed by three simple rules based on geometrical calculations, demonstrates flocking behaviour that is typical for flying birds. Without knowing about Reynolds's work, ornithologist Frank Heppner joined forces with mathematician Ulf Grenander and published the second computer model of bird flocking \cite{heppner:1990}. In their model the flock was a self-organizing collection of individuals, whose behaviour was based on stochastic nonlinear differential equations. Nevertheless, both models base their assumptions on common grounds and model the behaviour of individuals on the, at times contradictory, clues of attraction and repulsion. The mutual coexistence and importance for the congregation's structure of these two clues was already suggested by Okubo \cite{okubo:1980}.

After 1990 papers regarding computer models of bird flocking subsided. The rare exceptions were the studies of the evolution of flocking behaviour \cite{reynolds:1993a,reynolds:1993b,reynolds:1994,spector:2002,spector:2003} and Heppner's unpublished study of flock take-off and landing \cite{heppner:1997}. In the last few years the field has been slowly regaining scientific interest. Recent research, however, builds on the two original models. Tanner, Jadbabaie, and Pappas \cite{tanner:2003a,tanner:2003b} for example concentrate on the stability analysis of an organized flock that is based on Reynolds's model. Couzin \etal\ \cite{couzin:2005}, on the other hand, employ Reynolds's model to examine leadership and decision making in animal groups on the move. Their approach adds a preferred flight direction only to a proportion of the modelled digital birds. Their study reveals that the larger the group the smaller the proportion of informed individuals needed to guide the group, and that only a small proportion is required to achieve great accuracy. A rare example of a different, and also the most recent, approach is that of Wiltschko and Nehmzow \cite{wiltschko:2005}, but its primary concern is not flocking but rather the navigation process employed by pigeons. 

The primary reasons for the loss of research interest may lie hidden in the mathematical nature of these simulations as well as in the amount of work that is required to master the effects that parameter changes have on the displayed behaviour. Even Heppner and Grenander \cite{heppner:1990} admit that the interesting patterns were discovered serendipitously and that considerable trial-and-error experimentation was needed before flock-like behaviour was produced.

Furthermore, one can hardly imagine that flocking birds flying at speeds up to 150km/h \cite{heppner:1997} have the time or the ability to perform sophisticated or time-intensive mathematical calculations. Even Parrish \etal\ \cite{parrish:1997a} state that there must be simple traffic rules for species' engaging in collective movement. To continue, considering our perception of the surrounding environment, it is difficult to imagine that birds are able to perceive precise (\ie\ crisp) information (\eg\ distance). However, all of the existing mathematical models assume such capabilities.

Furthermore, the mathematical nature of the existing models means that a substantial mathematical understanding was required for their construction as is required for their thorough understanding. In my opinion this represents a major drawback for their usability. The mathematical nature, if truth be told, makes the models difficult to understand by the audience they were designed for. The latter is predominantly composed of ethologists, not mathematicians. Even if one makes the models as black-box modules and allows only changing the values of parameters, this would not suffice for truth-testing. Truth-testing any sort of simulation that purports to represent natural behaviour is extremely difficult, and has not often been done, especially in behaviour. Models are usually too crude, or have too many special conditions to be readily tested with real-world data. This is probably why ethologists have difficulties in using the models for testing the existing hypotheses or forming new ones.

%--
\subsection{Computer Model by Craig W. Reynolds}
\label{sec:birdFlocks:cwr}
Traditionally an animator who wanted to animate a bird flock would carefully set up numerous key frames that defined the motion paths of every single flock member. When animating a line flock, especially if it is a very small one, such an approach is somehow possible. Difficulties arise when large cluster flocks are being animated. In this case animating each and every flock member becomes painstaking and tedious, and is, disregarding the difficulty of corrections, without inter-bird collisions almost impossible to do. 

As already mentioned, Craig W. Reynolds, in his pioneering work from 1987, presented the first computer model of bird flocking \cite{reynolds:1987}. When reflecting on how to animate a bird flock he treated the latter as any group of entities that exhibit the general class of aligned, noncolliding, aggregate motion. This means that with the term flock Reynolds refers also to schools, herds, \etc\ However, when speaking about bird flocks it can be seen that his notion is more strict than definition~\ref{def:flock}. In fact, he assumes an organized flock to be coordinated in all of the flight parameters as well as that there are no collisions. However, the requirements of definition~\ref{def:flock} are met also when coordination is only in some of the parameters of flight and the definition does not make note of the absence of collisions. The latter on one hand seems plausible, however, on the other hand, especially in large cluster flocks, because of their size, the relatively small inter-bird spacing and fast manoeuvres, seems almost impossible.

In the mid 1980s the decentralization ideology \cite{resnick:1997} was becoming ever more influential. This was also the time when the research field of \emph{artificial life} \cite{adami:1998,emmenche:1994,langton:1989} was emerging. Both, together with Reynolds's prior work \cite{reynolds:1978,reynolds:1982}, led him toward the idea that a flock of birds, as perceived by one of its members, is something completely different than as perceived by an outside observer. It is much like the difference between driving in traffic and standing on a roadside watching traffic whiz by. This represented an important step forward. He did not look at the flock as a whole any more or searched for a single rule that describes it---known as \emph{top-down approach}. On the contrary, he imagined what it would be like to be a member of the flock and searched for the rules to follow in order to stay in the flock---known as the \emph{bottom-up approach}. This approach is characteristic for modelling artificial life \cite{adami:1998,emmenche:1994,gardner:1970,langton:1984,langton:1989,rucker:1993,terzopoulos:1994,terzopoulos:1999,tu:1999,ward:2001}. I am talking about modelling by constructing a large number of primary entities, whose local interactions base on simple rules, and observing the emergent global behaviour, behaviour not previously programmed according to specific rules \cite{emmenche:1994}. Later, in computer graphics, Reynolds's approach, when one actually seeks to model the behaviour of an object and not its shape or physical properties,  became known as \emph{behavioural animation} \cite{reynolds:1987}.

Reynolds \cite{reynolds:1987} came to the conclusion that as a member of a flock he would have to successfully coordinate three different drives. He found out that, in order to fly without collisions, he would have to make sure that he was not too close to any of his flockmates. In other words, he would try to maintain a certain separation distance. Furthermore, he would have to try to fly with the same flight speed and in the same flight direction as his flockmates. This also means that it would be highly unlikely that he collided with them in the near future. And finally if he noticed that all of the flockmates were on one of his sides he would wish to drift towards them. Written in the form of simple rules these drives are \cite{reynolds:1987,reynolds:1999,reynolds:2000}:

\begin{itemize}
\item {\emph{separation}}: avoid collisions with nearby flockmates,
\item {\emph{alignment}}: attempt to match flight speed and flight direction with nearby flockmates,
\item {\emph{cohesion}}: attempt to stay close to nearby flockmates.
\end{itemize}

The cohesion and separation drives together represent the so-called \emph{attraction-repul\-sion} scheme \cite{okubo:1980}. The alignment drive, on the other hand, is used to produce \emph{polarization}, which is another important feature of animal groups of uniform density \cite{parrish:1997a}. 

Reynolds translated the three drives to a set of geometrical equations, where he interpreted the expression `nearby flockmates' as the bird's immediate surroundings (see section~3.3.1). Actually, he found out that a bird does not require full knowledge about the positions, flight speed and flight direction of every bird in the flock, but only a small subset. The expression `nearby flockmates' thus addresses the bird's awareness of another bird and Reynolds based its computation on the distance and direction of the offset vector between them. His digital bird\footnote{Actually Reynolds refers to the digital (simulated) bird-like, ``bird-oid'' objects, generically as \emph{boids} even when they represent other sorts of creatures such as schooling fish \cite{reynolds:1987}.} thus actually has a localized perception of the world with a certain distance and field of view and can be visualized as a perception volume shaped like a sphere with a cone removed from the back. It is important to note that, when the digital birds are in a flock, the individual perception volumes overlap and each individual bird will probably end up in a number of perception volumes.

With a limited perception volume, Reynolds makes a very good point stating that a bird's perception of the world is severely limited by occlusion (\ie\ nearby birds hide those far away), but inside the perception volume he does not take this into account. Furthermore, even though he limits the digital bird's awareness of the world, the perceived information is accurate, meaning that the digital bird has full and precise knowledge about the position, flight speed and flight direction of its flockmates. In my opinion this approach is still defective. It is true that Reynolds does not try to model visual perception but tries to make available approximately the same information that is available to a bird as the end result of perceptual and cognitive processes. However, all of the obtained information is still based on visually perceptible information. A bird's visual perception is not limited only by occlusion, but also by the fact that the ability to sense distance, apart from being affected by the degree of binocular overlap, decreases with distance itself. Moreover, in his latest implementation of the model,\footnote{OpenSteer v0.8, \href{http://opensteer.sourceforge.net/}{http://opensteer.sourceforge.net/}.} Reynolds uses three distinct perception volumes, one per drive. According to my experiments (see Chapter~6) this introduces unwanted `unnatural' behaviour.

Furthermore, Reynolds models the cohesion drive as the digital bird's tendency to fly toward the centre of mass of the nearby flockmates. I find this somewhat questionable. The centre of mass is a mathematical construct and it is difficult to believe that a real bird has knowledge of such constructs or uses them to compute its action. It is true that it was Pliny \cite{heppner:1997} who noted that ``it is a peculiarity of the starling kind that they fly in flocks and wheel round in a sort of circular ball, all making towards the centre of the flock''. \sidenote{However, in my opinion, real birds might not have any idea about the centre of the flock and their making towards it might be just an emergent property by itself.}{v1.1.20050210 [FHH]: yeah, I tend to agree with this. In a really big flock, an interior bird might not have any idea where the center is.}

In Reynolds's approach an individual digital bird thus, based on the precise information about the position, flight speed and flight direction of its flockmates, using geometrical equations, computes the three desired changes of flight direction and flight speed, each satisfying one drive. As Reynolds models the digital bird as a point mass (simple) vehicle \cite{reynolds:1987,reynolds:1999}, he represents the desired change in flight direction and flight speed as the physical force that would induce it. The digital bird thus computes the actual change in flight direction and flight speed by computing a weighted sum of the resulting three physical forces (see section~3.3.1 for more detail).
 
%--
\subsection{Computer Model by Frank H. Heppner and Ulf Grenander}
\label{sec:birdFlocks:fhh}
As a contrast to animators who try to produce a flock animation that visually resembles a natural flock in a way that could take in a theatre spectator, flocks interest ornithologists from a completely different point of view. The synchrony, or apparent synchrony, in the turning movements of cluster flocks has drawn much attention from ornithologists. The key research directions were, and still are, guided mostly by the question how flocks coordinate their movement and decide when to wheel or turn. With respect to this in most investigations the leading role was played by the search for evidence that would answer the question of existence or necessity of a flock leader. Indeed, many researchers assumed a leader's existence and presumed it directed the movement of the whole flock \cite{heppner:1990}. This approach was also taken by Heppner and Hafner \cite{heppner:1974b}, but they proceeded to demonstrate the formidable obstacles to visual or acoustic communication between such a putative leader and its followers. Nevertheless, efforts to identify the flock's leader have been so far unsuccessful, even when using methods that permit an analysis of the position of individually identified birds in a free-flying pigeon flock \cite{pomeroy:1992}. In the meantime reports appeared \cite{davis:1980} which suggested that the coordinated turning might be an emergent phenomenon; the result of individual birds `voting' with their bodies on the flight path the flock should take. According to this theory the flock turns in the direction expressed by the initiators only when a `critical mass' is reached. Recently, a similar approach has been used by Couzin \etal\ \cite{couzin:2005} to examine leadership and decision making in animal groups on the move.

Similar to Reynolds, ornithologist Frank H. Heppner started pondering the idea that flocks might be an emergent phenomenon; the result of a group of individuals following simple rules. This idea was, again as in Reynolds's case, initially inspired by artificial life research; by Conway's `Game of Life' \cite{gardner:1970}, to be more precise. Moreover, also by the emerging suggestions that coordination in cluster flocks could be achieved by the application of nonlinear dynamics \cite{okubo:1986}. As ornithologists in most cases are not mathematicians, Heppner joined forces with applied mathematician Ulf Grenander, to translate his ideas into mathematical equations. The result of their work was the second computer model of bird flocking \cite{heppner:1990} given through a stochastic differential driven by a Poisson process. As in Reynolds's case, the model developed by Heppner and Grenander also suggested that birds coordinate through three, but different drives. They assumed that the coordination in flocks emerges because birds try to accomplish the following drives:

\begin{itemize}
\item \emph{homing}: each member of the flock tries to stay in the roosting area,
\item \emph{velocity regulation}: each member of the flock tries to fly with a certain predefined flight speed---it tries to return to that speed if perturbed,
\item \emph{interaction}: if two flockmates are too close to one another, they try to move apart; if they are too distant, they do not influence each other; otherwise they try to move closer together. 
\end{itemize}

As already said, Grenander translated these drives into mathematical equations that depended on the current position of the digital bird (homing), its flight speed (velocity regulation) or its distance from other digital birds (interaction). But their digital bird has complete and precise information about the locations of other digital birds. With this Heppner and Grenander make an unrealistic assumption because, as already discussed, real birds have limited and inaccurate perceptive capabilities.

As a contrast to Reynolds, who modelled the principles of attraction and repulsion in the form of two distinct drives, Heppner and Grenander combined them into one drive (interaction). Combined or not, the two drives seem reasonable and `natural', after all, their mutual coexistence and importance for the congregation's structure has already been suggested by Okubo \cite{okubo:1980}. On the other hand, the velocity regulation drive used by Heppner and Grenander seems somewhat strange. They derived this drive from aerodynamic theory; with a given power output and configuration, an aircraft will maintain a constant speed, it will return to that speed if perturbed. A real bird would not have to make a decision about this. In my opinion, this results in a questionable effect. Take, for example, a digital bird that slowed down because it was too close to one of its flockmates. It will speed up. But not because it is trying to catch up with the flock, but because it is returning to the predefined preferred flight speed. In my opinion this behaviour resembles more to an aggregation of birds that happen to be flying together than to a flock of birds that are trying to fly together. \sidenote{Furthermore, it is hard to grasp that a bird has a predefined preferred flight speed.}{v1.1.20050210 [FHH]: believe it---birds tend to fly (in straight lines) at very consistent speeds.} Even if it does, this cannot be constant in time. Even though birds (especially in line flocks) tend to fly in relatively straight lines and at very consistent speeds, they tend to change their flight speed regardless of the speed of the flock, which might, however, be primarily caused by fatigue or other distractions (\eg\ wind gusts). 

Another important feature of flocks is alignment. Heppner and Grenander \cite{heppner:1990} did not model it specifically, but they mention that in certain cases organized flocks maintaining straight direction of flight emerged. This might be caused by the perception model they used. Their digital bird has complete and accurate information about its surroundings, which means it has full and precise knowledge about the locations of all surrounding digital birds that are closer than a predefined distance. However, in their quest for realism, Heppner and Grenander, introduced also a special influence which was intended to simulate the effects of wind gusts and random local disturbances. They modelled it using a Poisson stochastic process and state that it was of crucial importance for achieving flock-like behaviour \cite{heppner:1990}.

 	\input{thesis-chap3_animat}
 	\input{thesis-chap4_fuzzymodelling}
 	\input{thesis-chap5_fuzzyanimat}
 	\input{thesis-chap6_analysis}
 	% !TeX root = ./thesis.tex










%==============================
\chapter{Conclusion}
\label{chap:conclusion}

%-----
\EBlettrine{The} phenomenon known as collective animal behaviour is one of the most beautiful spectacles one can observe in nature. Although researched and analysed by scientists from many disciplines and perspectives it is still puzzling in many ways. Examples of such puzzling questions are why did collective behaviour (especially highly organized forms) evolve and why do we see so much variation in behaviour even in closely related species. Through the construction of a genetic fuzzy system capable of evolving various forms of collective behaviour this study represents an attempt in shedding additional light on potential reasons why highly organized collective behaviour evolved.

We started the study by expanding on a known fuzzy model \cite{lebarbajec2005fuzzy,lebarbajec2005simulating} for the purpose of studying predator-prey dynamics in hand-crafted models. Prey animats in this model \cite{demsar2014simulated} could exhibit two types of behaviour -- a social one where they actively strived for grouping (via cohesion and alignment drives) and an individualistic one where they did not. We focused on vision as the principal means of perception, took into account occlusion \cite{kunz2012simulations} and working memory limitations \cite{ballerini2008interaction,engle1999individual,sherry1989hippocampus}, and considered three target selection (predation) tactics. With the first tactic predators attacked the nearest out of visible prey individuals, with the second the most visually isolated prey individual out of the visible ones, and with the third the centre of the visible group (visible prey individuals). To achieve biological relevance we tuned the parameters based on realistic data about birds (starlings, \emph{Sturnus vulgaris}, for prey, and peregrine falcon, \emph{Falco peregrinus}, for the predator). The study suggests that the most successful predator is the one that attacks the most visually isolated individual, while the least successful predator is the one that attacks the centre of the visible group, a result similar to those reported by field observations \cite{zoratto2010aerial}. As a plus, results obtained by our study suggest that, from a prey individual's perspective, social behaviour is more advantageous than individualistic behaviour, which strengthens our belief in the hypothesis that cluster flocking might serve as be a mechanism for protection from predation.

Field observations suggest that predators in nature are able to, at least partially, overcome the defensive benefits of prey grouping by using an assortment of sophisticated hunting strategies \cite{cresswell2011predicting,forsman1998visual,gazda2005division,handegard2012dynamics,hector1986cooperative,kane2014falcons,lopez2006bottlenose,nottestad2002digging,rutz2012predator}. As the parametrization and tuning of such tactics in a hand-crafted model would be a tiresome task we developed an evolutionary model that simulates the evolution of composite target selection tactics \cite{demsar2015simulating}. The most successful predators were those that first dived deep into the centre of the nearby prey group causing chaos and dispersal of the group. Following that they targeted stragglers (individuals that in the process got separated from the rest of the group). The tactic, which we termed as the dispersing tactic, is similar in function to the tactics used by several predators in nature \cite{larsson2012why,pavlov2000patterns}. In our study predators that used the dispersing tactic came out as significantly more successful hunters in a direct competition with predators that used a mixture of simple tactics. Again a result corroborating field observations \cite{pavlov2000patterns}. However, this was true only in the case when our model took into account the possibility of predator confusion. The concept of predator confusion is based on the confusability hypothesis, which suggests that a group of visually similar prey might make it difficult for the predator to select and track its target \cite{nishimura2002predator,zheng2005behavior,kunz2006prey,olson2013predator,olson2016evolution,rutz2012predator}. A different story was the case of the prey's delayed response, a defensive manoeuvre where prey rather then escaping on first sight of the predator, delay their response to a later point in time, and then try to outsmart the predator with rapid movement \cite{partridge1982structure}. The only predators able to, at least to some degree, overcome the defensive benefits of this escape manoeuvre, were again the predators that used the dispersing tactic. Because the dispersing tactic yields higher success to predators we can assume that dispersing the group reduces the group's defensive benefits. This strengthens our belief in the hypothesis that compact groups of prey might function as a defensive mechanism from predation. The absence of an advantage of the dispersing tactic over simple predation tactics when predator confusion is not at play indicates that predator confusion might have played an important role in the evolution of advanced predation tactics, as well. All of these findings were a clear indication of potential interplay between target selection tactics and the evolution of prey group behaviour.

Several studies already pursued the artificial evolution of collective animal behaviour, most by tuning parameters of previously presented non-evolutionary models. Very few succeeded to evolve it from scratch, and even in these cases the evolved behaviour can be termed as ``crude.'' Based on presented material the successful studies portray either clumping \cite{biswas2014causes,hein2015evolution,witkowski2016emergence}, or swarming with collisions \cite{olson2013predator,olson2015exploring,olson2016evolution,witkowski2016emergence}. To study how predation tactics influence the evolution of prey behaviour we designed a novel open-ended, artificial life-like evolutionary model where the drives of individual animats are encoded via linguistic fuzzy rules \cite{demsar2017evolution}. In our genetic fuzzy system prey and predator animats coexist in a shared environment. Based on knowledge about predator target selection tactics gained from our previous research \cite{demsar2014simulated,demsar2015simulating} we designed several types of hand-crafted predators that attack evolving prey. Subsequently, in our model only the survival instincts of prey animats steer the evolution of their behaviour and collective behaviour will emerge only if it will help prey animats survive. We analysed the evolved prey behaviour and showed that based on biologically relevant metrics \cite{couzin2002collective,vicsek2012collective,tunstrom2013collective} our evolutionary model is capable of producing a wide range of behaviours, some qualitatively and visually similar to those reported by experimental studies \cite{tunstrom2013collective}. Since we used a genetic fuzzy system we were able to further analyse the evolved behaviours by studying the fuzzy rule bases that govern the actions of individual animats. Doing so we showed that when clustering the rule bases by the type of evolved behaviour and observing the average proportion of rule antecedents that contain predator related linguistic variables there exists a statistically significant difference between the rule bases. This gives us confidence in advocating that artificial life-like evolutionary modelling based on linguistic fuzzy rule-based systems could be used for answering the illusive biological questions ``why'' collective animal behaviour evolved, and due to their linguistic nature also provide a deeper insight into the ``how.''

To gain further insight into potential ``whys'' we used the newly developed genetic fuzzy system in a controlled experiment where prey evolved while subject to multiple, systematically picked predation tactics simultaneously \cite{demsar2016balanced}. The predation tactics can be split into two groups; those for which the natural defensive response of prey might be grouping and those for which the natural response might be dispersing. We classified the evolved behaviours using quantitative metrics in a similar fashion as previous studies \cite{couzin2002collective,vicsek2012collective,tunstrom2013collective}. When prey evolved while exposed to predators that adopted tactics from only one group the results of evolution corroborated with previous studies \cite{biswas2014causes,olson2013predator,olson2016evolution,wood2007evolving}; prey animats evolved either grouping or dispersing behaviour, with values of metrics characteristic for milling or swarming. When prey animats evolved while exposed to antagonistic pressures that at the same time steered the evolution towards grouping and towards dispersing we detected a significant increase of polarization in motion of prey groups. This suggest that exposure to antagonistic predation pressures might be a necessary requirement for prey individuals to evolve parallel movement. This could indicate that the direction of evolution (grouping or dispersing) is not A versus B, but a labile result -- whether grouping or dispersing evolves depends on a) the nature of the group, and b) the pressures that the group finds itself facing.

\paragraph{Limitations of this study and future work} Throughout our research we devised a number of ideas which could potentially lead to interesting future studies of collective animal behaviour. When it comes to application of evolutionary models for help with providing answers to biological questions the most obvious research advances lie in upgrades towards a higher biological relevance. Evolutionary models are usually simplified due to high computational demands of genetic algorithms and as a result the models are most often restricted to two dimensions, animats in them have unrealistic perception systems, and animats traditionally move with constant speeds, etc. To allow the animats to vary their speed in an evolutionary model we would probably need to implement some kind of fatigue system as well, so that, just like in nature \cite{norin2016measurement,roche2013finding}, animats would not be able to move with their maximum speed indefinitely. 

Another possible direction would be the investigation of how heterogeneity influences the evolved behaviour. Some recent studies suggest that in an algorithm mimicking artificial evolution heterogeneous groups might evolve a different behaviour than homogeneous ones \cite{olson2015exploring}. Others suggest that heterogeneous groups might be necessary to achieve a more ``natural'' behaviour \cite{demsar2013family}, and that differences among individuals might be essential for group coordination \cite{marras2012information,marras2013schooling}. In nature heterogeneity (both in behaviour as well as in physiology) is always present, for example birds in a flock often differ in size, gender, age, and some times even species \cite{lebarbajec2009organized,jolles2013heterogeneous}. It is not uncommon that stronger members of the group are positioned at the ``safer'' parts of it \cite{hamilton1971geometry}, which leaves the weaker individuals more exposed to predator attacks. Some predators often intentionally target weak prey individuals \cite{domenici2014howsailfish,marras2015notsofast}, and by using models that consider also physiological heterogeneity one could, apart from studying its influence on prey behaviour, also study how it influences the adaptation of predator target selection tactics.

To our knowledge, in most of the existing models \cite{demsar2014simulated,demsar2015simulating,demsar2016balanced,demsar2017evolution,nishimura2002predator,zheng2005behavior}, the predator animat, once it selects its target, uses classical pursuit \cite{nahin2012chases} to chase its target. In nature some predators use advanced pursuit tactics, for example some species of falcons use the technique of motion camouflage \cite{kane2014falcons}. With this technique they either camouflage themselves against a fixed background object so that the targeted individual observes no relative motion between them and the fixed object, or they approach the targeted individual in a way that, from the targeted individual's point of view the predators always appear to be on the same bearing \cite{justh2006steering}. One possible future study might therefore be a genetic fuzzy system for the evolution of predator pursuit tactics. Or co-evolution of prey behaviour and predator pursuit tactics.

In nature predators often resort to group hunting \cite{creel1995communal,escobedo2014groupsize,fanshawe1993factors,lett2004continuous,muro2011wolfpack,packer1988evolution,scheel1991group}. Occasionally they cooperate (\ie cooperative hunting) to increase the probability of a successful hunting event \cite{creel1995communal,packer1988evolution}. Even though in our current model prey animats are often attacked by several predators at once the predators do not cooperate in any way. Studying the evolution of predator cooperation during hunting events would probably also lead to an interesting study.

Another possible upgrade would be the consideration of short term memory, which comes to play when, for example a predator moves out of view of the targeted individual (out of range or in its blind area). In current models, the targeted individual completely forgets that the predator was attacking it just a moment ago.

An important question is also flight initiation distance. In certain fish species prey as a defence mechanism delay their response \cite{partridge1982structure}. Our research already showed, that a delayed response is quite effective with certain target selection tactics. With an evolutionary model, however, we can study under what conditions (if) such a delayed reaction will emerge. Research in this this direction is already on its way, our current provisional results suggest that the answer might be related to the ratio between predator and prey speed.

The rule base is probably the most important part of a fuzzy animat since it defines the drives of the animat, which have the highest influence on the animat's behaviour. In genetic rule learning the data base of a fuzzy system is static, it does not evolve. As our genetic fuzzy system executes genetic rule learning, fuzzy variables, the linguistic terms, and the interpretation of logic connectives, which are all defined in the data base of a genetic fuzzy systems were hand-crafted. In our research this did not appear to be a limitation as our genetic fuzzy system is capable of evolving many of the forms of collective behaviour that can be commonly observed in nature.

The degree of truth for each fuzzy term is defined by its membership function, these functions come in many shapes (triangular, trapezoidal, singleton, Gaussian, etc.). Even though our algorithm supports many different types of membership functions we developed all models by means of trapezoidal/triangular functions only. These provide the lowest ratio between computational complexity and ease of conceptualization, visualization, and explanation. Again, in the case of our genetic fuzzy system the use of triangular functions did not seem to be a limiting factor as the repertoire of evolved collective behaviours is wider than in previous similar studies \cite{biswas2014causes,hein2015evolution,olson2013predator,olson2015exploring,olson2016evolution,reynolds1993evolved,sayers2009evolved,spector2003emergence,wood2007evolving}. Nevertheless, the evolution of rule bases with more sophisticated types of membership functions and the evolution of the whole fuzzy knowledge base seem like promising research directions for our future work.

To conclude, evolutionary models allowed us to untangle a number of interesting riddles related to collective behaviour already, but judging by the current trends we believe that the best is yet to come.
 	\begin{appendices}
		% !TeX root = ./thesis.tex
%==============================
\begin{razsirjeniPovzetek}

%-----
\section{Motivacija}
\lipsum[110-111]


%-----
\section{Demo poglavje A}
\lipsum[112-113]


%-----
\section{Demo poglavje B}
\lipsum[114-116]


%-----
\section{Demo poglavje C}
\lipsum[117-118]


%-----
\section{Demo poglavje D}
\lipsum[119-121]


%-----
\section{Rezultati}
\lipsum[122-124]


%-----
\section{Zaključek}

\lipsum[125-126] V pričujoči doktorski disertaciji so tako podani naslednji izvirni prispevki k znanosti:

\begin{itemize}
\item \emph{\lipsum[127][1-2]}
\item \emph{\lipsum[128][1-2]}
\item \emph{\lipsum[129][1-1]}
\end{itemize}


\end{razsirjeniPovzetek}

	\end{appendices}

\backmatter
 	% !TeX root = ./thesis.tex

% use bibtex for bibliography








%==============================
\bibliographystyle{pnas2011}
\bibliography{jd_PhD}


\end{document}
